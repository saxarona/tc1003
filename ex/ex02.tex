\documentclass{article}
%\usepackage[a4paper]{geometry}
\usepackage{fullpage}
\usepackage[utf8]{inputenc}
\usepackage[spanish, mexico]{babel}
\usepackage{lipsum}
\usepackage{bm}
\usepackage{upgreek}
\usepackage{enumitem}
\usepackage{mathrsfs}
\usepackage{amsmath}
\usepackage{amssymb}
\usepackage{tikz}
\usepackage{tcolorbox}
\usepackage{csquotes}
\usepackage{helvet}


% mathtools for: Aboxed (put box on last equation in align envirenment)
\usepackage{microtype} %improves the spacing between words and letters

%% COLOR DEFINITIONS

\usepackage{xcolor} % Enabling mixing colors and color's call by 'svgnames'

\definecolor{MyColor1}{rgb}{0.2,0.4,0.6} %mix personal color
\newcommand{\textb}{\color{Black} \usefont{OT1}{lmss}{m}{n}}
\newcommand{\blue}{\color{MyColor1} \usefont{OT1}{lmss}{m}{n}}
\newcommand{\blueb}{\color{MyColor1} \usefont{OT1}{lmss}{b}{n}}
\newcommand{\red}{\color{LightCoral} \usefont{OT1}{lmss}{m}{n}}
\newcommand{\green}{\color{Turquoise} \usefont{OT1}{lmss}{m}{n}}

\DeclareMathOperator{\trace}{trace}
\DeclareMathOperator{\diag}{diag}

%% FONTS AND COLORS

%    SECTIONS

\usepackage{titlesec}
\usepackage{sectsty}
%%%%%%%%%%%%%%%%%%%%%%%%
%set section/subsections HEADINGS font and color
%\sectionfont{\color{Black}}  % sets colour of sections
%\subsectionfont{\color{Black}}  % sets colour of sections

%set section enumerator to arabic number (see footnotes markings alternatives)
\renewcommand\thesection{\arabic{section}} %define sections numbering
\renewcommand\thesubsection{\thesection\arabic{subsection}} %subsec.num.

%define new section style
\newcommand{\mysection}{
\titleformat{\section} [runin] {\usefont{OT1}{lmss}{b}{n}\color{MyColor1}} 
{\thesection} {3pt} {} } 


% %	CAPTIONS
% \usepackage{caption}
% \usepackage{subcaption}
% %%%%%%%%%%%%%%%%%%%%%%%%
% \captionsetup[figure]{labelfont={color=Turquoise}}


%		!!!EQUATION (ARRAY) --> USING ALIGN INSTEAD
%using amsmath package to redefine eq. numeration (1.1, 1.2, ...) 
\renewcommand{\theequation}{\thesection.\arabic{equation}}

\setlength\parindent{0pt}




\makeatletter
\let\reftagform@=\tagform@
\def\tagform@#1{\maketag@@@{(\ignorespaces\textcolor{red}{#1}\unskip\@@italiccorr)}}
\renewcommand{\eqref}[1]{\textup{\reftagform@{\ref{#1}}}}
\makeatother
\usepackage{hyperref}
\hypersetup{colorlinks=true}

% For labeling top of page on every page but first one:
%\usepackage{fancyhdr}

\newcommand{\myclass}{TC1003 -- Matemáticas Discretas} % Class name?
\newcommand{\mytitle}{Examen Parcial 2} % Title of document?
\newcommand{\mydate}{29.04.2020} % The date?
\newcommand{\myheader}{
    \begin{flushleft}
        \large
        Nombre: \rule{10 cm}{0.4mm} \hfill Matrícula: \rule{2 cm}{0.4mm}\\[1.5ex]
    \end{flushleft}
}

\title{
    \myclass \\
    \textbf{\mytitle} \\
    \myheader
    \date{}
}

% You can set the date automatically by replacing "date goes here" with "\today"

% \renewcommand{\rmdefault}{phv} % Arial Font
\renewcommand{\familydefault}{\sfdefault}

% \pagestyle{fancy}
% \fancyhead{}
% \fancyhead[CO,CE]{{\small{{\bf{\mytitle}} -- \myclass}}}

\newcommand{\responserule}{{\large\rule{14 cm}{0.3mm}}}
\newcommand{\shortresponserule}{{\large\rule{5 cm}{0.3mm}}}
\newcommand{\veryshortresponserule}{{\large\rule{3 cm}{0.3mm}}}

\begin{document}
\maketitle

\vspace{-1.5cm}

{%
\footnotesize
\textit{Lee cuidadosamente y contesta lo que se te pide.
Este examen está pensado para resolverse de manera individual}

\textit{Al momento de contestar, intenta ser lo más explícito posible: se calificará con base en lo que esté escrito, y se considerará el proceso aún cuando la respuesta final esté errada.
Recuerda que puedes revisar material de la clase, el libro de texto o tus notas.
Buena suerte.}
}

\section{Lógica de Primer Orden (30\%)}

Convierte los siguientes enunciados a fórmulas de lógica de primer orden (15 \%)

\begin{enumerate}[label=\tt \alph*)]
    \itemsep0em
    \item \textit{Todas las hojas son del viento}
    \item \textit{Todos los invitados son campesinos}
    \item \textit{Algunos invitados plantan algodón}
    \item \textit{Otros invitados cultivan arroz}
    \item \textit{De los invitados, solamente yo planto salchichas empanizadas}
\end{enumerate}

Utilizando las identidades de dualidad de los cuantificadores, reescribe los enunciados anteriores:

\begin{enumerate}[label=\tt \alph*)]
    \item En lógica de primer orden (10 \%)
    \item En texto, nuevamente, basándote en las fórmulas del inciso anterior (5 \%)
\end{enumerate}

\section{Inferencia (45 \%)}

Los sistemas de recomendación de contenidos digitales se basan mayormente en modelos probabilísticos que entrenan con lo que los usuarios buscan y recomiendan como contenido similar.
Sin embargo, existe otro enfoque basado en razonamiento en el cual se generan reglas que analizan el contenido mismo, para decidir qué recomendaciones hacer si dos artistas contienen características similares.

Implementaremos la lógica detrás de un sistema basado en razonamiento para recomendación de grupos de K-Pop, considerando la siguiente información:

\vspace{2ex}

A los fans de Girls' Generation se les conoce como \textit{SONEs}, mientras que a los de Red Velvet les llaman \textit{Reveluvs} y a los de Blackpink, \textit{Blinks}.
A todos los \textit{SONEs} les gusta Red Velvet o Blackpink.
Si les gusta Red Velvet, entonces le gustan las baladas.
Si les gusta Blackpink, entonces les gusta el dance.
Si les gusta la música electrónica y son \textit{SONEs}, recomiéndales DJ Hyo.
La gente que disfruta el dance y las baladas, consideran a Chung Ha como una buena recomendación.
Todos los \textit{SONEs} a los que les gusta el drama, calificaron a Seohyun como una buena recomendación.
Taeyeon le interesa a todos los usuarios que gustan de las baladas y son \textit{SONEs}.
Si les gusta el drama y las baladas, es seguro que Heize les gustará.

\pagebreak

\begin{itemize}
    \item Genera la base de conocimiento proponiendo el vocabulario que usará tu sistema y convirtiendo las reglas a forma simbólica (5\%)
    \item Usando resolución por refutación, demuestra que si un usuario nuevo es fan de las Girls' Generation y es \textit{Reveluv}, le gustará Taeyeon (10\%)
    \item Considerando al mismo usuario, demuestra o refuta (\textit{prove or disprove}) que le gustará Heize (15\%)
    \item ¿Si un usuario es Sone, \textit{Reveluv}, \textit{Blink} y le gusta el drama, cuáles son las posibles recomendaciones que le hará el sistema? ¿Por qué? (15\%)
\end{itemize}

\section{Inducción y recursión (25 \%)}

\begin{enumerate}[label=\tt \alph*)]
    \item Define de manera recursiva (i.e. usando inducción) una función $Enigma(n)$ que calcule el producto de los primeros $n$ enteros positivos (10 \%)
    \item Demuestra que para todo $n \geq 4, Enigma(n) > 2^n$ (15 \%)
\end{enumerate}

\vfill

\textbf{De acuerdo con el Código de Ética del Tecnológico de Monterrey, mi desempeño en esta actividad estará guiado por la integridad académica.}
\end{document}