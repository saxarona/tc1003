\documentclass{article}
%\usepackage[a4paper]{geometry}
\usepackage{fullpage}
\usepackage[utf8]{inputenc}
\usepackage[spanish, mexico]{babel}
\usepackage{lipsum}
\usepackage{bm}
\usepackage{upgreek}
\usepackage{enumitem}
\usepackage{mathrsfs}
\usepackage{amsmath}
\usepackage{amssymb}
\usepackage{tikz}
\usepackage{tcolorbox}
\usepackage{csquotes}
\usepackage{helvet}


% mathtools for: Aboxed (put box on last equation in align envirenment)
\usepackage{microtype} %improves the spacing between words and letters

%% COLOR DEFINITIONS

\usepackage{xcolor} % Enabling mixing colors and color's call by 'svgnames'

\definecolor{MyColor1}{rgb}{0.2,0.4,0.6} %mix personal color
\newcommand{\textb}{\color{Black} \usefont{OT1}{lmss}{m}{n}}
\newcommand{\blue}{\color{MyColor1} \usefont{OT1}{lmss}{m}{n}}
\newcommand{\blueb}{\color{MyColor1} \usefont{OT1}{lmss}{b}{n}}
\newcommand{\red}{\color{LightCoral} \usefont{OT1}{lmss}{m}{n}}
\newcommand{\green}{\color{Turquoise} \usefont{OT1}{lmss}{m}{n}}

\DeclareMathOperator{\trace}{trace}
\DeclareMathOperator{\diag}{diag}

%% FONTS AND COLORS

%    SECTIONS

\usepackage{titlesec}
\usepackage{sectsty}
%%%%%%%%%%%%%%%%%%%%%%%%
%set section/subsections HEADINGS font and color
%\sectionfont{\color{Black}}  % sets colour of sections
%\subsectionfont{\color{Black}}  % sets colour of sections

%set section enumerator to arabic number (see footnotes markings alternatives)
\renewcommand\thesection{\arabic{section}} %define sections numbering
\renewcommand\thesubsection{\thesection\arabic{subsection}} %subsec.num.

%define new section style
\newcommand{\mysection}{
\titleformat{\section} [runin] {\usefont{OT1}{lmss}{b}{n}\color{MyColor1}} 
{\thesection} {3pt} {} } 


% %	CAPTIONS
% \usepackage{caption}
% \usepackage{subcaption}
% %%%%%%%%%%%%%%%%%%%%%%%%
% \captionsetup[figure]{labelfont={color=Turquoise}}


%		!!!EQUATION (ARRAY) --> USING ALIGN INSTEAD
%using amsmath package to redefine eq. numeration (1.1, 1.2, ...) 
\renewcommand{\theequation}{\thesection.\arabic{equation}}

\setlength\parindent{0pt}




\makeatletter
\let\reftagform@=\tagform@
\def\tagform@#1{\maketag@@@{(\ignorespaces\textcolor{red}{#1}\unskip\@@italiccorr)}}
\renewcommand{\eqref}[1]{\textup{\reftagform@{\ref{#1}}}}
\makeatother
\usepackage{hyperref}
\hypersetup{colorlinks=true}

% For labeling top of page on every page but first one:
%\usepackage{fancyhdr}

\newcommand{\myclass}{TC1003 -- Matemáticas Discretas} % Class name?
\newcommand{\mytitle}{Examen Parcial} % Title of document?
\newcommand{\mydate}{25.03.2020} % The date?
\newcommand{\myheader}{
    \begin{flushleft}
        \large
        Nombre: \rule{10 cm}{0.4mm} \hfill Matrícula: \rule{2 cm}{0.4mm}\\[1.5ex]
    \end{flushleft}
}

\title{
    \myclass \\
    \textbf{\mytitle} \\
    \myheader
    \date{}
}

% You can set the date automatically by replacing "date goes here" with "\today"

% \renewcommand{\rmdefault}{phv} % Arial Font
\renewcommand{\familydefault}{\sfdefault}

% \pagestyle{fancy}
% \fancyhead{}
% \fancyhead[CO,CE]{{\small{{\bf{\mytitle}} -- \myclass}}}

\newcommand{\responserule}{{\large\rule{14 cm}{0.3mm}}}
\newcommand{\shortresponserule}{{\large\rule{5 cm}{0.3mm}}}
\newcommand{\veryshortresponserule}{{\large\rule{3 cm}{0.3mm}}}

\begin{document}
\maketitle

\vspace{-1.5cm}

{%
\footnotesize
\textit{Lee cuidadosamente y contesta lo que se te pide.
Este examen está pensado para resolverse de manera individual}

\textit{Al momento de contestar, intenta ser lo más explícito posible: se calificará con base en lo que esté escrito, y se considerará el proceso aún cuando la respuesta final esté errada.
Recuerda que puedes revisar material de la clase, el libro de texto o tus notas.
Buena suerte.}
}

\section{Relaciones y funciones (15\%)}

Sean $A = \{1,2,3,4,5\}$ y $B = \{1,2,3,4,5\}$.
Para las siguientes relaciones, indica si son \textbf{reflexivas}, \textbf{simétricas} o \textbf{transitivas}.
Menciona también si son \textbf{funciones}. En caso de que lo sean, indica si son \textbf{totales} o \textbf{parciales}, y si son \textbf{inyecciones}, \textbf{sobreyecciones} o \textbf{biyecciones}.

\begin{enumerate}[label=\tt \alph*)]
    \itemsep0em
    \item $\{(1,1), (2,2), (3,3), (4,4)\}$
    \item $\{(2,2), (1,1), (3,3), (5,5), (4,4)\}$
    \item $\{(1,2), (2,1), (3,4), (4,3), (3,5), (5,3)\}$
    \item $\{(1,5), (2,3), (3,2), (4,4), (5, 4)\}$
    \item $A \times B$
\end{enumerate}


\section{Operaciones con conjuntos (15 \%)}

Calcula el resultado de las siguientes operaciones.

\begin{enumerate}[label=\tt \alph*)]
    \itemsep0em
    \item $\{1, 2, 3\} \cup \{z : z \in \mathbb{Z}, 4 \leq z < 10 \}$
    \item $\{1\} \times \{2,3,4\}$
    \item $|\mathscr{P}(\{n : n \in \mathbb{N} \cup \{0\}, n < 15\})|$
    \item $\{10, 20, 30\} \cap \{r : r \in \mathbb{N}\}$
    \item $\{1, 2, 3\} \cap \{4, 5, 6\}$
\end{enumerate}

\section{Lógica Proposicional (40 \%)}

El \textit{Biologic Space Lab} que orbita \textsc{SR388} sufrió una falla eléctrica dejando fuertes daños en algunos de sus sectores. Uno de los más afectados fue el que alberga muestras de la vida marina del planeta, el Sector 4 (AQA). Por ello, el equipo de mantenimiento te ha asignado la tarea de implementar un componente de control para el nivel del agua de los 32 tanques en el sector mientras se llevan a cabo las reparaciones.

Cada tanque tiene tres sensores: uno para la detección de movimiento (y así saber si en el tanque se encuentra algún espécimen o si el tanque se encuentra vacío), otro para medir la toxicidad del agua, y uno más para medir la cantidad de carga eléctrica, la cual es indicio de alguna falla. En el caso de los últimos dos, se marca como `peligroso' cuando se sobrepasa un umbral.

Los tanques envían actualizaciones al mecanismo de control una vez cada dos horas por medio de un packet de 8 bits, donde los primeros 3 corresponden a la lectura de los sensores (movimiento, toxicidad y carga, en ese orden), y los últimos 5 al ID del tanque (en binario).

HQ te ha notificado que la compuerta que alimenta a cada tanque debe cerrarse en cualquiera de las siguientes situaciones:

\begin{itemize}
    \itemsep0em
    \item Si las especies del tanque están a salvo (baja toxicidad y baja carga eléctrica).
    \item Si el tanque está deshabitado, pero es altamente tóxico, sin importar lo demás.
    \item Si el agua del tanque presenta una cantidad peligrosa de carga eléctrica, independientemente de que esté habitado o de su toxicidad.
\end{itemize}

Para el resto de los casos, se ahorrará energía y se dejará el flujo normal en cada tanque.

Diseña un sistema de control usando verificación de modelos para cerrar la compuerta de cada tanque cuando recibe un packet de información.
Sugerencias:

\begin{enumerate}[label=\tt \alph*)]
    \item Genera el alfabeto de tu sistema (4 \%)
    \item Expresa en simbología de lógica proposicional cada una de las condiciones de cierre (15 \%)
    \item Escribe la tabla de verdad completa del sistema (15 \%)
    \item Da tres ejemplos distintos de packets completos de información posibles y describe qué significa cada uno (6 \%)
\end{enumerate}

\section{Funciones y relaciones II (30 \%)}

Sea $M$ el conjunto de las notas musicales \textit{naturales} ($do, re, mi, \dots si$).

El \textit{círculo de quintas} es una relación que existe entre las notas, que sirve para determinar cuál será la armadura (los sostenidos o bemoles) en el pentagrama para una determinada pieza.
Esta relación se consigue tomando cada nota y agregándole su \textit{quinta}.
El proceso suele hacerse en orden, y se acostumbra a empezar desde la primera de las \textit{alteraciones}, que es $fa$.

La quinta $q$ de una nota $n$ no es más que sumarle 5 notas a la nota $n$ que tomamos como referencia, tomando en cuenta que en música se cuenta a la nota de referencia como la nota 1, e.g. que la \textit{primera} de una nota $n$ es ella misma, y que si $n = do$ entonces $q(do) = sol$.

\begin{enumerate}[label=\tt \alph*)]
    \item Escribe el círculo de quintas como un conjunto por \textbf{extensión} (10 \%)
    \item ¿Cuál es su cardinalidad? (2 \%)
    \item ¿Es reflexiva o irreflexiva? ¿Es transitiva o intransitiva? ¿simétrica o asimétrica? (3 \%)
    \item ¿Es el círculo de quintas una función? (2 \%)
    \item En dado caso de que lo sea, ¿es inyectiva? ¿suprayectiva? ¿es una bijyección? (3 \%)
    \item En caso de que no sea transitiva, haz que lo sea escribiendo la cerradura transitiva de $M$ (10 \%)\\
    {\footnotesize \textit{Hint: sugiero que sigas ordenadamente este procedimiento y trates de encontrar patrones que puedan servirte para deducir algunos resultados.}}
\end{enumerate}

\vfill

\textbf{De acuerdo con el Código de Ética del Tecnológico de Monterrey, mi desempeño en esta actividad estará guiado por la integridad académica.}
\end{document}