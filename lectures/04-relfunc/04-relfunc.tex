\documentclass[spanish, c]{beamer}

\usepackage[utf8]{inputenc}
\usepackage[spanish, mexico]{babel}
\usepackage{amsmath}
\usepackage{mathtools}
\usepackage{hyperref}
\usepackage{xcolor}
\usepackage{color}
\usepackage{ragged2e}
\usepackage{mathrsfs}
\usepackage{csquotes}
\usepackage{listings}
\usepackage[scaled]{beramono}
\usepackage[T1]{fontenc}
\usepackage{matlab-prettifier}
\usepackage{graphicx}
\usepackage{booktabs}
\usepackage{tikz}
\usepackage{venndiagram}

\renewcommand{\indent}{\hspace*{2em}}

% \usetikzlibrary{fit, shapes, arrows}

% \usepackage{courier}
% \usepackage{subfigure}
% \usepackage{enumerate}
% \usepackage{algorithmic}
% \usepackage{algorithm}

% \usepackage{listings}
% \usepackage{lstlinebgrd}

\usetheme{Boadilla}
\usefonttheme[onlymath]{serif}

\newcommand{\matlab}[1]{\lstinline[style=Matlab-editor]!#1!}
\newcommand\blfootnote[1]{%
\begingroup
\renewcommand\thefootnote{}\footnote{#1}%
\addtocounter{footnote}{-1}%
\endgroup
}

\lstset
{
    language = Matlab,
    style = Matlab-editor,
    basicstyle = \mlttfamily\scriptsize,
    escapechar = `,
    numbers = left,
    frame = tb,
}

\lstdefinestyle{output}
{
    language = {},
    basicstyle = \mlttfamily\scriptsize,
    escapechar = `,
    numbers = none,
    showtabs = false,
   	showstringspaces = false,
}

% Sets the templates
\definecolor{navyblue}{RGB}{0, 0, 128}
\definecolor{crimson}{RGB}{128, 16, 0}

\setbeamertemplate{navigation symbols}{}
\setbeamertemplate{headline}{}
\setbeamertemplate{title page}[default][colsep=-4bp,rounded=true]
\setbeamertemplate{footline}[frame number]
\setbeamertemplate{bibliography item}[text]
\setbeamertemplate{theorems}[numbered]

\setbeamercolor{title}{fg=navyblue, bg=white}
\setbeamercolor{frametitle}{fg=navyblue, bg=white}
\setbeamercolor{structure}{fg=navyblue}
\setbeamercolor{button}{fg=white,bg=navyblue}

\setbeamercovered{transparent}

\title{Comparando y asociando: relaciones y funciones}
\subtitle{Matemáticas Discretas \\ (TC1003)}
\author{
    \texorpdfstring{
        \begin{center}
            M.C. Xavier Sánchez Díaz \\
            \href{mailto:sax@tec.mx}{\texttt{sax@tec.mx}}
        \end{center}
    }
    {M.C. Xavier Sánchez Díaz}
}

\institute[Tecnológico de Monterrey]{\includegraphics[scale=0.5]{../img/logo}}
\date{}

\begin{document}

\setlength{\rightskip}{0pt}

\begin{frame}[plain]
    \titlepage        
\end{frame}

\begin{frame}{Outline}
    \tableofcontents
\end{frame}

\section{Relaciones}

\begin{frame}{Tuplas}{Relaciones}
    Una \alert{tupla} es una estructura matemática \textbf{de tamaño definido} y donde \textbf{el orden importa}. \pause

    \bigskip

    \begin{exampleblock}{Ejemplo de tupla}
        $$(1, 2), (2, 3), (3, 5)$$
    \end{exampleblock} \pause

    \bigskip
    
    ¿Cuál es el conjunto de celdas de un tablero de \textit{Battleship} si las casillas van de la $A-J$ y del $1-10$?

\end{frame}

\begin{frame}{Producto Cartesiano}{Relaciones}
    El \alert{producto Cartesiano} es el \textbf{conjunto} de todos los posibles valores que se pueden formar a partir de la combinación de dos conjuntos, de la siguiente manera: \pause

    \bigskip

    \begin{block}{Producto Cartesiano}
        $$A \times B = \{(a, b) : a \in A, b \in B\}$$
    \end{block} \pause

    \bigskip
    
    El conjunto de las casillas de un tablero de \textit{Battleship} es claramente el producto Cartesiano del conjunto de sus renglones y sus columnas.
\end{frame}

\begin{frame}{Definición de Relación}{Relaciones}
    Una \alert{relación} entre dos conjuntos $A$ y $B$ es cualquier conjunto $R$ de tal manera que $R \subseteq A \times B$. \pause

    \bigskip

    \begin{itemize}
        \item Piensa en el conjunto de alumnos presentes en el salón como $A$ \pause
        \item Piensa ahora en el conjunto de sillas disponibles en el salón como $B$ \pause
        \item ¿Cuál es el producto Cartesiano $A \times B$? \pause
        \item ¿Qué relación $R$ podemos formar sobre $A$ y $B$?
    \end{itemize} \pause

\end{frame}

\subsection{Cuantificadores y operaciones}

\begin{frame}{Cuantificadores}{Cuantificadores y operaciones}
    Claramente, las relaciones aplican \textit{para algunos Skywalker} y no para todos. \pause
    Sin embargo, también tenemos relaciones que podrían aplicar \textit{a todos}. \pause

    \begin{block}{Cuantificadores}
        \begin{itemize}
            \item $\forall$ que significa \textit{para todos}
            \item $\exists$ que significa \textit{existe}
            \item $\exists!$ que significa \textit{existe un único}
        \end{itemize}
    \end{block} \pause

    \bigskip

    Puedes agregarle negación frente a cada uno para cambiar el significado a lo contrario.
    ¿Qué significa la negación de cada uno de ellos?
\end{frame}

\begin{frame}{Relaciones inversas}{Relaciones}
    La \alert{inversa} $R^-1$ de una relación $R$ es \dots $R$ al revés.

    \bigskip
    
    \begin{block}{Relación inversa}
        Una relación inversa es cualquier relación $R^{-1}$ tal que
        $$R^{-1} = \{(b, a) : (a, b) \in R\}$$
    \end{block} \pause

    \bigskip

    \begin{itemize}
        \item Piensa en $A$ como el conjunto de \textit{Los Skywalker} \pause
        \item ¿Puedes hacer una relación de parentezco $R$ sobre $A^2$? \pause
        \item ¿Cuál sería la relación inversa?
    \end{itemize}

\end{frame}

\begin{frame}{Imagen de una relación}{Operaciones}
    La \alert{imagen} de una relación $R$ (usualmente denotada por $I$) es el conjunto de todos aquellos elementos $b$, es decir\dots \pause

    \bigskip

    \begin{block}{Imagen}
        $$I(R) = \{b : (a, b) \in R\}$$
    \end{block} \pause

    \bigskip

    ¿Cuál es la imagen en la relación $Padre$ en \textit{Los Skywalker}?
\end{frame}

\subsection{Propiedades de las relaciones}

\begin{frame}{Reflexividad}{Propiedades de las relaciones}
    
    \begin{block}{Reflexividad}
        $R$ es \alert{reflexiva} si y sólo si $\forall a \in A \left( \exists (a, a) \in R \right)$
    \end{block} \pause

    \bigskip

    \begin{itemize}
        \item Piensa en un ejemplo de una relación reflexiva \textit{(Hint: piensa en números)}
        \item Lo opuesto a la reflexividad es la \textbf{irreflexividad}: si para todos NO se cumple la condición
        \item Una relación puede no ser ni reflexiva ni irreflexiva
    \end{itemize}

\end{frame}

\begin{frame}{Transitividad}{Propiedades de las relaciones}

    \begin{block}{Transitividad}
        $R$ es \alert{transitiva} si y sólo si
        $$\forall(a,b) \in R\left((a,b) \in R \wedge (b,c) \in R \implies (a,c) \in R\right)$$
    \end{block} \pause

    \bigskip

    \begin{itemize}
        \item Piensa en un ejemplo de relación transitiva
        \item Lo opuesto a la transitividad es la \textbf{intransitividad}: si para todos NO se cumple la condición
        \item Una relación puede no ser ni transitiva ni intransitiva
    \end{itemize}

\end{frame}

\begin{frame}{Simetría}{Propiedades de las relaciones}

    \begin{block}{Simetría}
        $R$ es \alert{simétrica} si y solo si
        $$\forall(a,b) \in R\left( (a, b) \in R \implies (b, a) \in R \right)$$
    \end{block} \pause
    
    \bigskip

    \begin{itemize}
        \item Piensa en un ejemplo de relación simétrica
        \item Lo opuesto a la simetría es la \textbf{asimetría}: si para todos NO se cumple la condición
        \item Una relación puede no ser ni simétrica ni asimétrica
    \end{itemize}

\end{frame}

\begin{frame}{Relaciones de equivalencia}{Propiedades de las relaciones}

    \begin{block}{Equivalencia}
        Una relación $R$ es \alert{equivalente} si es \textbf{reflexiva}, \textbf{transitiva} y \textbf{simétrica}.
    \end{block} \pause

    \bigskip

    ¿Habías pensado que el $=$ es un operador que relaciona dos números de manera \textit{equivalente}?

\end{frame}

\subsection{Partición, Órdenes y Cerraduras}

\begin{frame}{Partición}{Partición, Órdenes y Cerraduras}
    
    Una \alert{partición} de $A$ es cualquier conjunto $B_{i \in I}$ de subconjuntos de $A$ que:

    \begin{itemize}
        \item No están vacíos
        \item Son disjuntos entre sí
        \item La unión generalizada de ellos cubre totalmente a $A$
    \end{itemize} \pause

    \bigskip

    Una \textit{repartición} de dulces a un conjunto de bolsas es justamente una partición del conjunto de dulces.

\end{frame}

\begin{frame}{Antisimetría}{Partición, Órdenes y Cerraduras}

    \begin{block}{Antisimetría}
        Una relación es \alert{antisimétrica} si y solo si
        $$\forall (a,b) \in R\left( (a, b) \in R \implies (b, a) \not \in R \right)$$
    \end{block} \pause

    \bigskip

    A una relación que es \textbf{reflexiva}, \textbf{transitiva} y \textbf{antisimétrica} se le conoce como \alert{orden parcial}, o \textit{poset}.

\end{frame}

\begin{frame}{Orden total}{Partición, Órdenes y Cerraduras}

    \begin{block}{Completez en una reflexión}
        Una relación \textbf{reflexiva} es \alert{completa} si y solo si
        $$\forall (a, b) \in A \left((a,b) \in R \vee (b, a) \in R \right)$$
    \end{block} \pause

    Cuando un \textit{poset} es completo (o lineal), se le conoce como \alert{orden total}. \pause

    \bigskip

    \begin{center}
        \Huge $$\leq \text{ vs } < $$
    \end{center}
\end{frame}

\begin{frame}{Cerraduras}{Partición, Órdenes y Cerraduras}
    La \alert{cerradura} (\textit{closure} en inglés) de $A$ bajo la relación $R$ (denotada por $R[A]$) es un \textbf{conjunto} del tamaño mínimo necesario para cumplir con la aplicación de $R$ a cada elemento de $A$, y tal que $A \subseteq R[A]$ \dots

    \pause

    \bigskip

    \begin{exampleblock}{Ejemplo de cerradura}
        \begin{itemize}
            \item \textbf{Q}: ¿Cuál es la \textbf{cerradura transitiva} de $A = \{(2, 3), (3, 4), (1, 2), (3, 1), (1,7),(7,8)\}$?
            \item \textbf{A}: \begin{align*}
                A = \{& (2, 3), (3, 4), (1, 2), (3, 1), (1,7), (7,8), \\
                      & (2, 4), (1, 3), (3, 2), (1, 8), (1, 4), (1, 1), (3, 3)\}
            \end{align*}
        \end{itemize}        
    \end{exampleblock}

    

\end{frame}

\section{Funciones}

\subsection{Mapeos y operaciones}

\subsection{Propiedades de las funciones}

\subsection{Equipotencia y el Principio del Palomar}

% What is a function
% Partial vs Full functions
% Domain
% Range
% Composition
% Injection
% Surjection
% Bijection
% Equinumerosity
% Pigeonhole Principle



% \section*{Referencias}

% \begin{frame}[t]{Referencias}
    % \nocite{bibID01}
    % \nocite{bibID02}

    % \bibliographystyle{IEEE}
    % \bibliography{biblio}
% \end{frame}

\end{document}