\documentclass[spanish, c]{beamer}

\usepackage[utf8]{inputenc}
%\usepackage[spanish, mexico]{babel}
\usepackage{amsmath}
\usepackage{mathtools}
\usepackage{hyperref}
\usepackage{xcolor}
\usepackage{color}
\usepackage{ragged2e}
\usepackage{mathrsfs}
\usepackage{csquotes}
\usepackage{listings}
\usepackage[scaled]{beramono}
\usepackage[T1]{fontenc}
\usepackage{matlab-prettifier}
\usepackage{graphicx}
\usepackage{booktabs}

\renewcommand{\indent}{\hspace*{2em}}

% \usepackage{tikz}

% \usetikzlibrary{fit, shapes, arrows}

% \usepackage{courier}
% \usepackage{subfigure}
% \usepackage{enumerate}
% \usepackage{algorithmic}
% \usepackage{algorithm}

% \usepackage{listings}
% \usepackage{lstlinebgrd}

\usetheme{Boadilla}
\usefonttheme[onlymath]{serif}

\newcommand{\matlab}[1]{\lstinline[style=Matlab-editor]!#1!}
\newcommand\blfootnote[1]{%
\begingroup
\renewcommand\thefootnote{}\footnote{#1}%
\addtocounter{footnote}{-1}%
\endgroup
}

\lstset
{
    language = Matlab,
    style = Matlab-editor,
    basicstyle = \mlttfamily\scriptsize,
    escapechar = `,
    numbers = left,
    frame = tb,
}

\lstdefinestyle{output}
{
    language = {},
    basicstyle = \mlttfamily\scriptsize,
    escapechar = `,
    numbers = none,
    showtabs = false,
   	showstringspaces = false,
}

% Sets the templates
\definecolor{navyblue}{RGB}{0, 0, 128}
\definecolor{crimson}{RGB}{128, 16, 0}

\setbeamertemplate{navigation symbols}{}
\setbeamertemplate{headline}{}
\setbeamertemplate{title page}[default][colsep=-4bp,rounded=true]
\setbeamertemplate{footline}[frame number]
\setbeamertemplate{bibliography item}[text]
\setbeamertemplate{theorems}[numbered]

\setbeamercolor{title}{fg=navyblue, bg=white}
\setbeamercolor{frametitle}{fg=navyblue, bg=white}
\setbeamercolor{structure}{fg=navyblue}
\setbeamercolor{button}{fg=white,bg=navyblue}

\setbeamercovered{transparent}

\title{Trabajando con la verdad}
\subtitle{Matemáticas Discretas \\ (TC1003)}
\author{
    \texorpdfstring{
        \begin{center}
            M.C. Xavier Sánchez Díaz \\
            \href{mailto:sax@tec.mx}{\texttt{sax@tec.mx}}
        \end{center}
    }
    {M.C. Xavier Sánchez Díaz}
}

\institute[Tecnológico de Monterrey]{\includegraphics[scale=0.5]{../img/logo}}
\date{}

\begin{document}

\setlength{\rightskip}{0pt}

\begin{frame}[plain]
    \titlepage        
\end{frame}

\begin{frame}{Outline}
    \tableofcontents
\end{frame}

\section{De los símbolos al significado}

\begin{frame}{Repaso}{De los símbolos al significado}
    Hasta ahora hemos revisado lo siguiente:

    \begin{itemize}[<+->]
        \item La conjunción (Te quiero con limón \alert{y} con sal)
        \item La disyunción (Coca \alert{o} Pepsi; cualquiera de las dos está bien)
        \item La negación (\alert{No} es cierto que ella te odie)
        \item La implicación (\alert{Si} te aplicas, \alert{entonces} vas a pasar el semestre sin problemas)
    \end{itemize}

    \bigskip \pause

    Además, también revisamos

    \begin{itemize}[<+->]
        \item Variables: $P, Q, R$
        \item Estatutos atómicos $P$ o estatutos compuestos $P \implies Q$
        \item Conectivos y operadores lógicos: $\vee, \wedge, \implies, \iff, \neg$
    \end{itemize}
\end{frame}

\begin{frame}{Las tablas de verdad I}{De los símbolos al significado}
    Para poder entender de dónde sale una \alert{tabla de verdad}, primero hay que entender cómo afecta cada operador a cada estatuto. \pause

    \bigskip

    Sabemos que si tenemos $P \wedge Q$ significa que tenemos que cumplir con ambas condiciones para decir la verdad.
    ¿Qué pasa en el siguiente enunciado? \pause

    \bigskip

    \[\neg P \wedge Q\] \pause

    ¿Cuál de las siguientes preguntas es la que representa al enunciado anterior? \pause

    \begin{itemize}[<+->]
        \item No tengo hambre y estoy enojado
        \item Ni tengo hambre, ni estoy enojado
        \item Tengo hambre y no estoy enojado
        \item No es cierto que tenga hambre y esté enojado
    \end{itemize} \pause

    ¿A qué está afectando el $\neg$?
\end{frame}

\begin{frame}{Aridad}{De los símbolos al significado}

    El término \alert{aridad} hace referencia a \textbf{cuántos posibles valores} puede tomar \textit{algo}. \pause

    \bigskip

    Como ejemplo, tenemos el operador $\vee$, que es un operador \alert{binario} (\textit{bi} de 2), puesto a que necesita 2 átomos para operar: \pause

    \bigskip

    \begin{center}
        \alert<4>{Esto} o \alert<5>{esto otro} \pause
    \end{center}

    \bigskip

    \visible<6->{
        \begin{itemize}
            \item ¿Qué otros operadores conoces que sean \textbf{binarios}?
            \item ¿Cuántos átomos necesita el $\neg$ para operar?
            \item ¿Cuántos valores posibles puede tomar una variable \textbf{atómica}?
        \end{itemize}
    }

\end{frame}

\begin{frame}{Las tablas de verdad II}{De los símbolos al significado}

    Una \alert{tabla de verdad} es una manera \textit{sencilla} de recordar cómo funcionan los conectivos lógicos. \pause

    \bigskip

    Dado a que las variables \textbf{atómicas} (también llamadas variables \textit{booleanas}) son \textbf{binarias}\dots \pause
    
    \begin{itemize}[<+->]
        \item ¿Cuántos posibles \textit{outcomes} tenemos para una sola variable $P$?
        \item ¿Cuántos outcomes tenemos para $P \wedge Q$?
        \item ¿De cuántas posibles \textit{maneras} podemos llegar a los posibles outcomes de $P \wedge Q$?
    \end{itemize} \pause

    \bigskip
    
    El número de renglones de una tabla de verdad \textbf{siempre es par}.

\end{frame}

\section{Detalle de tablas de verdad}

\begin{frame}{$P$}{Detalle de tablas de verdad}

    La tabla de verdad de $P$ consta de todos sus posibles valores: \pause

\begin{table}[H]
    \begin{tabular}{@{}l@{}}
    \toprule
    $P$ \\ \midrule
    T \\
    F \\ \bottomrule
    \end{tabular}
\end{table} \pause

\begin{itemize}[<+->]
    \item Tiene \textbf{dos} renglones puesto a que es \textbf{una sola variable} que puede tomar \textbf{dos valores}
    \item Equivale a la \textbf{presencia} o \textbf{ausencia} de una señal: \textit{prendido} o bien \textit{apagado}
    \item Ejemplo: \textit{Está lloviendo}
\end{itemize}
\end{frame}

\begin{frame}{$\neg P$}{Detalle de tablas de verdad}

    La tabla de verdad de $\neg P$ consta de los posibles valores de $P$ y por consiguiente, los valores de $\neg P$, que es lo contrario a eso: \pause

\begin{table}[H]
    \begin{tabular}{@{}cc@{}}
    \toprule
    $P$ & $\neg P$ \\ \midrule
    T   & F        \\
    F   & T        \\ \bottomrule
    \end{tabular}
\end{table} \pause

\begin{itemize}[<+->]
    \item ¿Cuántos renglones tiene?
    \item Equivale \textit{lo contrario} de lo que sea que valga $P$
    \item Ejemplo: \textit{\textbf{no es cierto que} está lloviendo}
\end{itemize}
\end{frame}

\begin{frame}{$P \vee Q$}{Detalle de las tablas de verdad}

    La tabla de verdad de $P \vee Q$ consta de los posibles valores de $P$, de $Q$ y de $P \vee Q$:


\begin{table}[H]
    \begin{tabular}{@{}ccc@{}}
    \toprule
    $P$ & $Q$ & $P \vee Q$ \\ \midrule
    T   & T    & T          \\
    T   & F    & T          \\
    F   & T    & T          \\
    F   & F    & F          \\ \bottomrule
    \end{tabular}
\end{table}

\begin{itemize}
    \item 4 renglones
    \item Ejemplo: \textit{Coca o pepsi}
    \item Con cualquier que lleves, ya cumpliste. La única manera de no cumplir es que no lleves ninguna de las dos.
\end{itemize}
\end{frame}

\begin{frame}{$P \wedge Q$}{Detalle de las tablas de verdad}

    La tabla de verdad de $P \wedge Q$ consta de los posibles valores de $P$, de $Q$ y de $P \wedge Q$:
    
    \begin{table}[H]
        \begin{tabular}{@{}ccc@{}}
        \toprule
        $P$ & $Q$ & $P \wedge Q$ \\ \midrule
        T   & T    & T          \\
        T   & F    & F          \\
        F   & T    & F          \\
        F   & F    & F          \\ \bottomrule
        \end{tabular}
    \end{table}

    \begin{itemize}
        \item 4 renglones
        \item Ejemplo: \textit{Sin leche y sin azúcar}
        \item La única manera de cumplir es si ambas son ciertas.
    \end{itemize}

\end{frame}

\begin{frame}{$P \implies Q$}{Detalle de las tablas de verdad}

    La tabla de verdad de $P \implies Q$ consta de los posibles valores de $P$, de $Q$ y de $P \implies Q$:
    
    \begin{table}[H]
        \begin{tabular}{@{}ccc@{}}
        \toprule
        $P$ & $Q$ & $P \implies Q$ \\ \midrule
        T   & T    & T          \\
        T   & F    & F          \\
        F   & T    & T          \\
        F   & F    & T          \\ \bottomrule
        \end{tabular}
    \end{table}

    \begin{itemize}
        \item 4 renglones
        \item Ejemplo: \textit{Si estudias, pasarás sin problemas el semestre}
        \item Si la hipótesis es verdadera, y no ocurre lo que dices, entonces mentías
        \item Si la hipótesis es falsa, puedo concluir cualquier cosa
    \end{itemize}

\end{frame}

\section{Equivalencias y leyes}

\begin{frame}{Jerarquía de operaciones lógicas}{Equivalencias y leyes}
    
    En orden de fuerza de cohesión, tenemos lo siguiente:

    \begin{enumerate}[<+->]
        \item $\neg$
        \item $\wedge, \vee$
        \item $\implies$
        \item $\iff$
    \end{enumerate} \pause

    Es decir que

    $$P \wedge Q \implies \neg R$$

    Si ($P$ y $Q$ son ciertas) entonces  (No es cierto que ($R$)).

\end{frame}

\begin{frame}{Equivalencias}{Equivalencias y leyes}

    Dos enunciados $P$ y $Q$ son \alert{equivalentes} ($P \equiv Q$) si sus tablas de verdad son iguales. \pause

    \bigskip

    \begin{enumerate}
        \item Construye la tabla de verdad de $P \implies Q$
        \item Construye la tabla de verdad de $Q \implies P$
        \item ¿Qué obtienes con la \textbf{conjunción} de los enunciados 1 y 2?
    \end{enumerate}

\end{frame}

\begin{frame}{¿Cómo sé qué es cierto y qué no?}{Equivalencias y leyes}

    Si $P$ y $Q$ son variables de verdad, $T$ es siempre verdadero y $F$ es siempre falso, entonces \dots \pause

    \begin{itemize}[<+->]
        \item $P \implies Q \equiv \neg P \vee Q$
        \item $\neg(\neg P) \equiv P$
        \item $Q \vee P \equiv P \vee Q$
        \item $Q \wedge P \equiv P \wedge Q$
        \item $\neg(P \vee Q) \equiv \neg P \wedge \neg Q$
        \item $\neg(P \wedge Q) \equiv \neg P \vee \neg Q$
        \item $(P \implies Q) \wedge (Q \implies P) \equiv P \iff Q$
        \item $P \vee P \equiv P$
        \item $P \wedge P \equiv P$
        \item $P \vee T \equiv T$
        \item $P \wedge T \equiv P$
        \item $P \vee F \equiv P$
        \item $P \wedge F \equiv F$
    \end{itemize}

\end{frame}

% What is control flow
% why is it important
% does it exist in math?
% how to represent it
% how to represent it in matlab
% practical cases

% \section*{Referencias}

% \begin{frame}[t]{Referencias}
    % \nocite{bibID01}
    % \nocite{bibID02}

    % \bibliographystyle{IEEE}
    % \bibliography{biblio}
% \end{frame}

\end{document}