\documentclass[spanish, c]{beamer}

\usepackage[utf8]{inputenc}
\usepackage[spanish, mexico]{babel}
\usepackage{amsmath}
\usepackage{mathtools}
\usepackage{hyperref}
\usepackage{xcolor}
\usepackage{color}
\usepackage{ragged2e}
\usepackage{mathrsfs}
\usepackage{csquotes}
\usepackage{listings}
\usepackage[scaled]{beramono}
\usepackage[T1]{fontenc}
\usepackage{matlab-prettifier}
\usepackage{graphicx}
\usepackage{booktabs}
\usepackage{tikz}
\usepackage{venndiagram}
\usepackage{semantic}

\renewcommand{\indent}{\hspace*{2em}}

% \usetikzlibrary{fit, shapes, arrows}

% \usepackage{courier}
% \usepackage{subfigure}
% \usepackage{enumerate}
% \usepackage{algorithmic}
% \usepackage{algorithm}

% \usepackage{listings}
% \usepackage{lstlinebgrd}

\usetheme{Boadilla}
\usefonttheme[onlymath]{serif}

\newcommand{\matlab}[1]{\lstinline[style=Matlab-editor]!#1!}
\newcommand\blfootnote[1]{%
\begingroup
\renewcommand\thefootnote{}\footnote{#1}%
\addtocounter{footnote}{-1}%
\endgroup
}

\lstset
{
    language = Matlab,
    style = Matlab-editor,
    basicstyle = \mlttfamily\scriptsize,
    escapechar = `,
    numbers = left,
    frame = tb,
}

\lstdefinestyle{output}
{
    language = {},
    basicstyle = \mlttfamily\scriptsize,
    escapechar = `,
    numbers = none,
    showtabs = false,
   	showstringspaces = false,
}

% Sets the templates
\definecolor{navyblue}{RGB}{0, 0, 128}
\definecolor{crimson}{RGB}{128, 16, 0}

\setbeamertemplate{navigation symbols}{}
\setbeamertemplate{headline}{}
\setbeamertemplate{title page}[default][colsep=-4bp,rounded=true]
\setbeamertemplate{footline}[frame number]
\setbeamertemplate{bibliography item}[text]
\setbeamertemplate{theorems}[numbered]

\setbeamercolor{title}{fg=navyblue, bg=white}
\setbeamercolor{frametitle}{fg=navyblue, bg=white}
\setbeamercolor{structure}{fg=navyblue}
\setbeamercolor{button}{fg=white,bg=navyblue}

\setbeamercovered{transparent}

\title{La prueba matemática}
\subtitle{Matemáticas Discretas \\ (TC1003)}
\author{
    \texorpdfstring{
        \begin{center}
            M.C. Xavier Sánchez Díaz \\
            \href{mailto:mail@tec.mx}{\texttt{mail@tec.mx}}
        \end{center}
    }
    {M.C. Xavier Sánchez Díaz}
}

\institute[Tecnológico de Monterrey]{\includegraphics[scale=0.5]{../img/logo}}
\date{}

\begin{document}

\setlength{\rightskip}{0pt}

\begin{frame}[plain]
    \titlepage        
\end{frame}

\begin{frame}{Outline}
    \tableofcontents
\end{frame}

\section{Introducción y vocabulario}

\begin{frame}{La prueba matemática}{Introducción y Vocabulario}

    En matemáticas, un \alert{teorema} es una oración que es \textbf{verdadera}. \pause

    \bigskip

    Una \alert{prueba matemática} (\textbf{prueba}, de ahora en adelante bajo este contexto) es una \textbf{secuencia} de oraciones que forman un argumento para \textbf{demostrar} que un \textbf{teorema} es cierto. \pause

    \bigskip

    Las oraciones en una prueba incluyen \textbf{axiomas}, \textbf{hipótesis} del teorema que queremos demostrar, y teoremas previamente probados.
\end{frame}

\begin{frame}{Probamos a partir de la verdad}{Introducción y Vocabulario}

    Para poder \textbf{decidir} si una oración es \textbf{verdadera} necesitamos un marco de referencia sobre lo que está permitido hacer, y sobre lo que \textbf{ya sabemos que es verdad}, como los axiomas. \pause

    \bigskip
    
    Para eso existen los \alert{sistemas de prueba o deducción}. Por ejemplo, el sistema de pruebas Hilbert $\mathrm{H}$, utiliza \textit{Modus Ponens} como regla de inferencia, y sus axiomas son específicamente:

    \begin{itemize}
        \item $\vdash (A \implies (B \implies A))$
        \item $\vdash (A \implies (B \implies C)) \implies ((A \implies B) \implies (A \implies C))$
        \item $\vdash (\neg B \implies \neg A) \implies (A \implies B)$
    \end{itemize} \pause

    \alert{Todas} aquellas fórmulas válidas son resultado de una deducción formal usando estas reglas, y por eso se dice que es un sistema de deducción \alert{completo}. \pause

    \bigskip

    De manera contraria se tiene que \alert{solamente} aquellas fórmulas válidas son deducibles usando estas reglas, y por eso se dice que es un sistema de deducción \alert{sólido}.
\end{frame}

\section{Técnicas y Recomendaciones Generales}

\begin{frame}{Para demostrar\dots}{Recomendaciones generales}

    \begin{itemize}[<+->]
        \item Es importante leer y entender completamente la oración que queremos probar (quizá sea lo más difícil en algunas ocasiones)
        \item Puede haber teoremas compuestos que podemos partir en teoremás más pequeños
        \item Intenta revisar rápidamente algunos casos simples del teorema para comprobar si es cierto o no
        \item \textbf{Escribe} la prueba---sólo si la hacemos por escrito podemos demostrar su \textbf{solidez}
    \end{itemize}

\end{frame}

\begin{frame}{Prueba directa}{Técnicas de prueba}

    Son aquellas pruebas donde la aplicación del teorema tal cual nos da el \textit{camino} que buscamos. \pause

    \begin{theorem}
        0 es un número par.
    \end{theorem}

    \begin{proof}
        La definición de un número par es aquél que puede ser dividido entre 2, obteniendo un resultado entero y un residuo de 0.

        Al dividir $\dfrac{0}{2} = 0$, tenemos un resultado entero (0) y un residuo de 0, y por tanto es un número par.
    \end{proof}

\end{frame}

\begin{frame}{Prueba directa}{Técnicas de Prueba}
    
    \begin{theorem}
        \label{th:02}
        Si $n$ es un entero positivo impar, entonces $n^2$ es también impar.
    \end{theorem}

    \begin{proof}
        Podemos describir cualquier entero positivo impar como $n = 2k + 1$ para cualquier número entero $k \geq 0$. Entonces,

        $$n^2 = (2k +1)^2 = 4k^2 +4k + 1 = 2(2k^2 + 2k) + 1$$

        Como $2(2k^2 + 2k)$ es par, y cualquier número par más uno nos da un número impar, entonces $n^2$ es impar.
    \end{proof}

\end{frame}


\begin{frame}{Prueba constructiva}{Técnicas de Prueba}

    Son pruebas donde hay que \textbf{construir} o \textbf{generar} un objeto con \textit{cierta propiedad} para revisar la existencia del mismo.
    
    \begin{theorem}
        Si $X \subseteq A \cap B$, entonces $X \subseteq A$ y $X \subseteq B$.
    \end{theorem}

    \begin{proof}
        Sea $x$ un elemento del conjunto $X$, de tal modo que $x \in X$.
        Como sabemos que $x \in X$, y que $X \subseteq A \cap B$, entonces significa que $x \in A \cap B$.

        Y dado a que $x \in A \cap B$, significa que $x \in A$ y $x \in B$.
        Y como $x \in X$, entonces $X \subseteq A$, y de igual manera $X \subseteq B$.
    \end{proof}

\end{frame}

\begin{frame}{Prueba por contradicción}{Técnicas de Prueba}

    Sea $S$ un estatuto verdadero. Si mostramos que $\neg S \implies falso$ es cierto, entonces es suficiente para mostrar que $S$ es, como dijimos en un principio, verdadero. \pause

    \begin{theorem}
        Sea $n$ un entero positivo. Si $n^2$ es par, entonces $n$ es par.
    \end{theorem}

    \begin{proof}
        Por contradicción, asumimos que $n^2$ es par pero que $n$ es impar.
        
        Como $n$ es impar, sabemos del Teorema~\ref{th:02} que $n^2$ debe ser impar. Sin embargo, esto genera una contradicción, porque asumimos que $n^2$ es par.
    \end{proof}

\end{frame}

\begin{frame}{Contraejemplo}{Técnicas de Prueba}

    Si nos dicen que algo es verdad, y encontramos algún caso en el que no lo sea, entonces es falso (que es básicamente lo mismo que hacemos en contradicción). \pause

    $$\forall a \in \mathbb{R}, \forall b \in {R} (a^2 = b^2 \implies a = b)$$ \pause

    \bigskip

    \begin{proof}
        Sean $a = 1$ y $b = -1$. Entonces, $a^2 = 1$ y también $b^2 = 1$, sin embargo $a \neq b$. Por tanto, el estatuto inicial es falso.
    \end{proof}

\end{frame}

\begin{frame}{Inducción matemática I}{Técnicas de Prueba}

    Para cada entero positivo $n \in \mathbb{N}$, sea $P(n)$ un estatuto matemático que depende de $n$.
    Queremos probar que $P(n)$ es cierto para cualquier entero positivo $n$. Una prueba por \alert{inducción} se lleva a cabo entonces de la siguiente forma: \pause

    \begin{enumerate}
        \item Probamos que $P(1)$ es verdadero.\pause
        \item Probamos ahora que para todo $n > 1$, si $P(n)$ es cierto, entonces $P(n+1)$ es también cierto \pause
    \end{enumerate}

    El primer paso se conoce como el \alert{caso base}, mientras que al paso dos se le llama \alert{paso de inducción}.
    En el paso de inducción, escogemos un valor arbitrario para $n$, y asumimos que $P(n)$ es verdad (como nos sugieren). A esto se le llama \alert{la hipótesis de inducción}.
\end{frame}

\begin{frame}{Inducción matemática II}{Técnicas de Prueba}

    \begin{theorem}
        Para todos los enteros positivos $n$, $1 + 2 + 3 + \dots + n = \dfrac{n(n+1)}{2}$
    \end{theorem}

    \bigskip

    Comenzamos con $n=1$. Si $n=1$ entonces tenemos $1 = \dfrac{1(1+1)}{2}$, lo cual es correcto.
    
    Para el paso de inducción, sea $n \geq 1$, y asumamos que el teorema es cierto para $n$.
    Si es cierto para $n$, entonces debe ser cierto para $n+1$:

    $${\color{blue} 1 + 2 + 3 + \dots + n} + (n+1) = \frac{(n+1)(n+2)}{2}$$
    
    Dado que el lado azul del igual mantiene la propiedad, entonces podemos reemplazarlo por $\frac{n(n+1)}{2}$.
\end{frame}

\begin{frame}{Inducción matemática II (cont.)}{Técnicas de Prueba}

    $${\color{blue} 1 + 2 + 3 + \dots + n} + (n+1) = {\color{magenta} \frac{(n+1)(n+2)}{2}}$$

        Si reemplazamos lo azul por la propiedad de $\frac{n(n+1)}{2}$, entonces
        $$\frac{n(n+1)}{2} + (n+1) = \frac{n^2 + n}{2} + (n+1)$$
        y con esto podemos expandir a
        $$\frac{n^2 + 3n + 2}{2}$$
        para luego factorizar como
        $$\color{magenta} \frac{(n+1)(n+2)}{2}$$

        que es la propiedad que estamos buscando $\quad \square$
\end{frame}

\section{Recursión}

\begin{frame}{¿Por qué funciona la inducción matemática?}{Técnicas de Prueba}
    
    La inducción matemática \textit{sólo funciona} en los números naturales. \pause

    \bigskip

    Esto se debe a que $\mathbb{N}$ tiene una \alert{estructura ordenada} que puede ser simplificada en los dos pasos de la inducción: \pause
    
    \begin{itemize}
        \item Existe un caso base (el 1) \pause
        \item El siguiente número es \textit{obtenible} como el número \textit{actual} + 1 (sucesor) \pause
    \end{itemize}

    Y esto da como resultado que se le pueda aplicar \alert{recursión}.
\end{frame}

\begin{frame}{Recursión}
    La \alert{recursión} es una propiedad de \textit{algunas} \textbf{estructuras matemáticas} las cuales pueden ser expresadas en la forma que revisamos (caso base + regla general). \pause

    \bigskip

    \begin{center}
        ¿Qué funciones se te vienen a la mente que puedan ser expresadas en este formato?
    \end{center}
\end{frame}

% \section*{Referencias}

% \begin{frame}[t]{Referencias}
    % \nocite{bibID01}
    % \nocite{bibID02}

    % \bibliographystyle{IEEE}
    % \bibliography{biblio}
% \end{frame}

\end{document}