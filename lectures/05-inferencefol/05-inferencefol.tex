\documentclass[spanish, c]{beamer}

\usepackage[utf8]{inputenc}
\usepackage[spanish, mexico]{babel}
\usepackage{amsmath}
\usepackage{mathtools}
\usepackage{hyperref}
\usepackage{xcolor}
\usepackage{color}
\usepackage{ragged2e}
\usepackage{mathrsfs}
\usepackage{csquotes}
\usepackage{listings}
\usepackage[scaled]{beramono}
\usepackage[T1]{fontenc}
\usepackage{matlab-prettifier}
\usepackage{graphicx}
\usepackage{booktabs}
\usepackage{tikz}
\usepackage{venndiagram}

\renewcommand{\indent}{\hspace*{2em}}

% \usetikzlibrary{fit, shapes, arrows}

% \usepackage{courier}
% \usepackage{subfigure}
% \usepackage{enumerate}
% \usepackage{algorithmic}
% \usepackage{algorithm}

% \usepackage{listings}
% \usepackage{lstlinebgrd}

\usetheme{Boadilla}
\usefonttheme[onlymath]{serif}

\newcommand{\matlab}[1]{\lstinline[style=Matlab-editor]!#1!}
\newcommand\blfootnote[1]{%
\begingroup
\renewcommand\thefootnote{}\footnote{#1}%
\addtocounter{footnote}{-1}%
\endgroup
}

\lstset
{
    language = Matlab,
    style = Matlab-editor,
    basicstyle = \mlttfamily\scriptsize,
    escapechar = `,
    numbers = left,
    frame = tb,
}

\lstdefinestyle{output}
{
    language = {},
    basicstyle = \mlttfamily\scriptsize,
    escapechar = `,
    numbers = none,
    showtabs = false,
   	showstringspaces = false,
}

% Sets the templates
\definecolor{navyblue}{RGB}{0, 0, 128}
\definecolor{crimson}{RGB}{128, 16, 0}

\setbeamertemplate{navigation symbols}{}
\setbeamertemplate{headline}{}
\setbeamertemplate{title page}[default][colsep=-4bp,rounded=true]
\setbeamertemplate{footline}[frame number]
\setbeamertemplate{bibliography item}[text]
\setbeamertemplate{theorems}[numbered]

\setbeamercolor{title}{fg=navyblue, bg=white}
\setbeamercolor{frametitle}{fg=navyblue, bg=white}
\setbeamercolor{structure}{fg=navyblue}
\setbeamercolor{button}{fg=white,bg=navyblue}

\setbeamercovered{transparent}

\title{Expansiones a la lógica: Lógica de Primer Orden e Inferencia}
\subtitle{Matemáticas Discretas \\ (TC1003)}
\author{
    \texorpdfstring{
        \begin{center}
            M.C. Xavier Sánchez Díaz \\
            \href{mailto:sax@tec.mx}{\texttt{sax@tec.mx}}
        \end{center}
    }
    {M.C. Xavier Sánchez Díaz}
}

\institute[Tecnológico de Monterrey]{\includegraphics[scale=0.5]{../img/logo}}
\date{}

\begin{document}

\setlength{\rightskip}{0pt}

\begin{frame}[plain]
    \titlepage        
\end{frame}

\begin{frame}{Outline}
    \tableofcontents
\end{frame}

\section{Recap de Lógica}

\begin{frame}{Lógica proposicional}{Recap de Lógica}
    Previamente, analizamos un poco \textit{la verdad} en algunas oraciones.

    \begin{itemize}[<+->]
        \item Aprendimos la diferencia entre un \textbf{estatuto} y una oración.
        \item También aprendimos a identificar cuando un estatuto era \textbf{atómico}.
        \item Revisamos la diferencia entre los operadores \textbf{binarios} y un operador \textbf{unitario}
        \item Aprendimos también la simbología necesaria:
        \bigskip
        \begin{itemize}[<+->]
            \item \textbf{Conjunción}: $\wedge$
            \item \textbf{Disyunción}: $\vee$
            \item \textbf{Negación}: $\neg$
            \item \textbf{Implicación}: $\implies$
            \item \textbf{Doble implicación}: $\iff$
            \item \textbf{Disyunción exclusiva}: $\oplus$
        \end{itemize}
    \end{itemize}
\end{frame}

\section{Lógica de Primer Orden}

\begin{frame}{Expresividad}{Lógica de Primer Orden}
    \begin{itemize}
        \item El sol sale por el este
        \item Cinco ballenas mueren al día
        \item Si hay de sirloin, me traes cinco.
    \end{itemize} \pause

    \begin{center}
        ¿Qué tienen en común estas proposiciones?
    \end{center} \pause

    \begin{itemize}
        \item Todas son proposiciones que hablan de \textbf{un solo valor de verdad}.
        \item La veracidad en ellas es \textit{absoluta} y se presenta de manera \textit{aislada}.
    \end{itemize}

\end{frame}

\begin{frame}{Expresividad}{Lógica de Primer Orden}

    Pensemos en el siguiente ejemplo: \textit{de noche, todos los gatos son pardos}.
    
    ¿Cómo la reescribimos? en una forma más fácilmente 'expresable' con lo que hemos visto? \pause

    \bigskip

    \textit{Si es de noche, entonces todos los gatos son pardos}, para que quede en la forma $ P \implies Q$
    donde $P =$ \textit{es de noche} y $Q = $ \textit{todos los gatos son pardos}. \pause

    \bigskip

    Tendríamos que pensar en \textit{todos los gatos} como \textbf{un solo objeto} para que esto funcione con la lógica que conocemos. \pause

    \bigskip

    Pongámosle números, en donde nuestro universo contempla los 3 gatos que---esperamos aún---viven en el Campus.
    ¿Cómo hacemos para describir que al menos dos son pardos durante la noche?

\end{frame}

\begin{frame}{Expresividad}{Lógica de Primer Orden}
    \textit{Si es de noche, entonces al menos dos de los tres gatos que viven en el Campus son pardos.} \pause
    $$P \implies Q$$
    donde $P$ es lo mismo: \textit{es de noche}, y $Q$ cambió: \textit{al menos dos de los tres gatos que viven en el Campus son pardos}. \pause
    
    \bigskip

    Claramente, si queremos expresar algo con cantidades o condiciones adicionales, alguna de las fórmulas atómicas debe \textit{absorber} esta información. Significa que van a haber cosas que \textbf{no podremos expresar} de esta manera general. \pause

    A esta capacidad (o incapacidad) de expresar se le conoce como \textbf{poder expresivo} o \alert{expresividad}. La lógica proposicional es de expresividad limitada.
    
    Sin embargo, podemos \textit{expandirla} para poder expresar otro tipo de cosas.
\end{frame}

\begin{frame}{Cuantificadores}{Lógica de Primer Orden}

    Recordemos ahora los cuantificadores que vimos al hablar del tema de relaciones y funciones: \pause

    \begin{itemize}[<+->]
        \item Cuantificador universal: $\forall$ que significa \textit{para todos}
        \item Cuantificador existencial: $\exists$ que significa \textit{existe} (o sea, para al menos uno)
        \item Cuantificador de unicidad: $\exists!$ que significa \textit{existe únicamente uno} (o sea, para solamente uno) y es un caso especial del cuantificador existencial
    \end{itemize}

    \pause

    Con esto podemos acercarnos un poco más al ejemplo de los gatos que necesitamos: \pause

    \textit{Para todo $x$, si $x$ es un gato, y es de noche, entonces $x$ es pardo}.
    $$\forall x(Gx \implies Px) \quad \text{ o bien } \forall x(G(x) \implies P(x))$$ \pause
    
    ¿Qué son $G$ y $P$?
\end{frame}

\begin{frame}{Relaciones, Funciones y Predicados}{Lógica de Primer Orden}

    La \alert{lógica de primer orden} (LPO o FOL por sus siglas en inglés) trabaja con \textbf{cuantificadores} y \textbf{relaciones y funciones} para tener un mayor poder expresivo. \pause

    \bigskip

    Podemos pensar en el predicado $G(x)$ o $Gx$ como una función unitaria de la forma $G \colon \mathscr{V} \to \mathscr{T}$ donde $\mathscr{V}$ es el conjunto de posibles \textit{variables} en nuestra fórmula, y $\mathscr{T}$ son los posibles valores de verdad de cada una de ellas---cierto, o falso.
    Bajo ese concepto, entonces $G(x)$ puede pensarse como la función \textit{$x$ es un gato} que puede ser verdadero o falso. \pause

    $Px$ significa entonces que $x$ es pardo.
\end{frame}

\begin{frame}{Implicaciones para prácticamente todo}{Lógica de Primer Orden}
    
    Pensemos en otro ejemplo felino: \textit{los $L$eones y los $T$igres son $P$eligrosos}. ¿Cómo expresamos esto en lógica de primer orden?
    \pause
    
    $\forall x((Lx \vee Tx) \implies Px)$ o bien $\forall x(Lx \implies Px) \wedge \forall x(Tx \implies Px)$ que podemos leer literalmente como

    \begin{itemize}
        \item Para todo $x$, si $x$ es un león o un tigre, entonces $x$ es peligroso
        \item Para todo $x$, si $x$ es un león entonces es peligroso. Y además, para todo $x$, si $x$ es un tigre entonces es peligroso.
    \end{itemize} \pause

    No podríamos agrupar $\forall x (Lx \wedge Tx)$ porque esto significaría que $x$ es un tigre y también un león, y lo que estaríamos diciendo tendría que ser verdad para todos aquellos $x$ que son tigres-leones.
\end{frame}

\begin{frame}{Más ejemplos}{Lógica de Primer Orden}
    \begin{itemize}[<+->]
        \item \textit{Algunos compositores son poetas} $\Rightarrow \exists x (Cx \wedge Px)$
        \item \textit{Todos aman a alguien} $\Rightarrow \forall x \exists y (Lxy)$
        \item \textit{Existe un número primo menor a 7} $\Rightarrow \exists x(Px \wedge (x < 7))$
    \end{itemize} \pause

    \bigskip

    ¿Cómo expresaríamos lo siguiente? \pause

    \bigskip
    
    \begin{itemize}
        \item \textit{Todos los hombres hablan más que Charles Chaplin} \pause
        \item \textit{Si un triángulo tiene un ángulo recto, entonces no es equilátero}
    \end{itemize}    
\end{frame}

\section{Formalización de la lógica}

\begin{frame}{Dualidad de los cuantificadores}{Formalización de la lógica}

    Como el Ying Yang, existe cierta \alert{dualidad} de los \textbf{cuantificadores} \pause
    \bigskip
    \begin{itemize}[<+->]
        \item $\neg \forall x(\alpha) \equiv \exists x(\neg \alpha)$
        \item $\neg \exists x(\alpha) \equiv \forall x(\neg \alpha)$
        \item $\forall x(\alpha) \equiv \neg \exists x(\neg \alpha)$
        \item $\exists x(\alpha) \equiv \neg \forall x(\neg \alpha)$
    \end{itemize} \pause

    El caso especial $\exists! x(\alpha)$ hace referencia a $\exists x[\alpha \wedge \forall y(\alpha \implies x=y)]$ \pause

    si no existe algo que cumpla, significa que para todos no se cumple algo \dots
    $$\forall x\neg[\alpha \wedge \forall y(\alpha \implies x=y)]$$ \pause
    \dots lo que implica que usando DeMorgan y las reglas de arriba, logramos
    $$\forall x [\neg \alpha \vee \exists y(\neg \alpha)]$$

\end{frame}

\begin{frame}{Interpretación}{Formalización de la lógica}
    En lógica de primer orden hablamos de fórmulas. Una fórmula $A$ tiene distintos predicados (como $Gx$) y distintas constantes (como Charles Chaplin). \pause

    Una \alert{interpretación} $\mathscr{I}_A$ de $A$ es una tripleta $(D, \{R_1, \dots, R_m\}, \{d_1, \dots, d_k\})$ donde

    \begin{itemize}
        \item $D$ es un dominio \textit{no-vacío}
        \item $R_i$ es una relación $n_i$-aria sobre $D$ que se asigna al $n_i$-ario predicado de la fórmula $A$
        \item $d_i \in D$ es asignado a la constante $a_i$.
    \end{itemize} \pause

    Por ejemplo, para la fórmula $A = \forall x p(a,x)$, tres de sus posibles interpretaciones pueden ser
        $$\mathscr{I}_1 = (\mathbb{N}, \{\leq\}, \{0\}) \quad \,
        \mathscr{I}_2 = (\mathbb{N}, \{\geq\}, \{5\}) \quad \,
        \mathscr{I}_3 = (\mathbb{Z}, \{\leq\}, \{0\})$$
    \end{frame}

\begin{frame}{Asignaciones y valores de verdad}{Formalización de la lógica}

    Teniendo una interpretación $\mathscr{I}_A$, podemos tener una \alert{asignación} $\sigma_{\mathscr{I}_a} \colon \mathscr{V} \to D$ que mapea toda variable libre $v \in \mathscr{V}$ a un elemento $d \in D$. \pause

    Esta asignación tiene distintos valores de verdad, dependiendo de qué variables se utilicen. Este valor de verdad se denota como
    $$v_{\sigma_{\mathscr{I}_A}}(A)$$ \pause
    y se lee como \textit{el valor de verdad de la fórmula $A$ bajo la interpretación $\mathscr{I}_A$ y la asignación $\sigma_{\mathscr{I}_A}$}. \pause

    ¿Cuántos posibles valores tiene $v_{\sigma_{\mathscr{I}_A}}(A)$?
\end{frame}


\begin{frame}{Validez y Factibilidad}{Formalización de la lógica}
    Toda esta información nos da las herramientas necesarias para poder entender la \textit{validez} de una fórmula $A$ de lógica de primer orden:

    \bigskip

    \begin{itemize}
        \item $A$ es \alert{verdad} en $\mathscr{I}$ (o $\mathscr{I}$ es un \alert{modelo} para $A$) si y solo si $v_{\mathscr{I}}(A) = T$. La notación que usaremos es $\mathscr{I} \models A$
        \item $A$ es \alert{válida} is \textbf{para toda} interpretación $\mathscr{I}$, $\mathscr{I} \models A$. La notación que usaremos es $\models A$.
        \item $A$ es \alert{factible} (\textit{satisfiable}) si para alguna interpretación $\mathscr{I}$, $\mathscr{I} \models A$.
        \item $A$ es \alert{no factible} (\textit{unsatisfiable}) si no es factible (duh) .
        \item $A$ es \alert{falsificable} (\textit{falsifiable}) si no es válida.
    \end{itemize}
\end{frame}

% How to do that

% What is a function
% Partial vs Full functions
% Domain
% Range
% Composition
% Injection
% Surjection
% Bijection
% Equinumerosity
% Pigeonhole Principle



% \section*{Referencias}

% \begin{frame}[t]{Referencias}
    % \nocite{bibID01}
    % \nocite{bibID02}

    % \bibliographystyle{IEEE}
    % \bibliography{biblio}
% \end{frame}

\end{document}