\documentclass[spanish, c]{beamer}

\usepackage[utf8]{inputenc}
\usepackage[spanish, mexico]{babel}
\usepackage{amsmath}
\usepackage{mathtools}
\usepackage{hyperref}
\usepackage{xcolor}
\usepackage{color}
\usepackage{ragged2e}
\usepackage{mathrsfs}
\usepackage{csquotes}
\usepackage{listings}
\usepackage[scaled]{beramono}
\usepackage[T1]{fontenc}
\usepackage{matlab-prettifier}
\usepackage{graphicx}
\usepackage{booktabs}
\usepackage{tikz}
\usepackage{venndiagram}
\usepackage{semantic}

\renewcommand{\indent}{\hspace*{2em}}

% \usetikzlibrary{fit, shapes, arrows}

% \usepackage{courier}
% \usepackage{subfigure}
% \usepackage{enumerate}
% \usepackage{algorithmic}
% \usepackage{algorithm}

% \usepackage{listings}
% \usepackage{lstlinebgrd}

\usetheme{Boadilla}
\usefonttheme[onlymath]{serif}

\newcommand{\matlab}[1]{\lstinline[style=Matlab-editor]!#1!}
\newcommand\blfootnote[1]{%
\begingroup
\renewcommand\thefootnote{}\footnote{#1}%
\addtocounter{footnote}{-1}%
\endgroup
}

\lstset
{
    language = Matlab,
    style = Matlab-editor,
    basicstyle = \mlttfamily\scriptsize,
    escapechar = `,
    numbers = left,
    frame = tb,
}

\lstdefinestyle{output}
{
    language = {},
    basicstyle = \mlttfamily\scriptsize,
    escapechar = `,
    numbers = none,
    showtabs = false,
   	showstringspaces = false,
}

% Sets the templates
\definecolor{navyblue}{RGB}{0, 0, 128}
\definecolor{crimson}{RGB}{128, 16, 0}

\setbeamertemplate{navigation symbols}{}
\setbeamertemplate{headline}{}
\setbeamertemplate{title page}[default][colsep=-4bp,rounded=true]
\setbeamertemplate{footline}[frame number]
\setbeamertemplate{bibliography item}[text]
\setbeamertemplate{theorems}[numbered]

\setbeamercolor{title}{fg=navyblue, bg=white}
\setbeamercolor{frametitle}{fg=navyblue, bg=white}
\setbeamercolor{structure}{fg=navyblue}
\setbeamercolor{button}{fg=white,bg=navyblue}

\setbeamercovered{transparent}

\title{Deducción e Inferencia}
\subtitle{Matemáticas Discretas \\ (TC1003)}
\author{
    \texorpdfstring{
        \begin{center}
            M.C. Xavier Sánchez Díaz \\
            \href{mailto:sax@tec.mx}{\texttt{sax@tec.mx}}
        \end{center}
    }
    {M.C. Xavier Sánchez Díaz}
}

\institute[Tecnológico de Monterrey]{\includegraphics[scale=0.5]{../img/logo}}
\date{}

\begin{document}

\setlength{\rightskip}{0pt}

\begin{frame}[plain]
    \titlepage        
\end{frame}

\begin{frame}{Outline}
    \tableofcontents
\end{frame}

\section{El proceso de inferencia}

\begin{frame}{Constructos}{Inferencia}
    Ya sabemos \textit{construir} oraciones lógicas.
    El siguiente paso ahora es \alert{operar} con estas oraciones usando un vocabulario propio de la inferencia:

    \begin{itemize}[<+->]
        \item Premisa
        \item Conclusión
        \item Teorema
        \item Prueba
        \item Axioma
        \item Sistema de deducción
    \end{itemize}
\end{frame}

\begin{frame}{Sistema de deducción}{Inferencia}
    
    \begin{block}{Definición}
        Un \alert{sistema deductivo} es un conjunto de fórmulas llamados \textbf{axiomas} y una serie de \textbf{reglas de inferencia}.
    \end{block} \pause

    \bigskip

    \begin{itemize}
        \item \alert{Axiomas}: fórmulas que se asume que son ciertas desde el principio---hechos. Son la base de todas las demás reglas. \pause
        \item \alert{Reglas de inferencia}: fórmulas que ayudan a generar nuevas reglas a partir de los axiomas.
    \end{itemize}

\end{frame}

\begin{frame}{Prueba}{Inferencia}

    \begin{block}{Definición}
    Una \alert{prueba} en un \textbf{sistema deductivo} es una secuencia de fórmulas $S = \langle A_1, \dots , A_n \rangle$ de tal modo que cada fórmula $A_i$ es o un \textbf{axioma} o puede ser inferida usando fórmulas previas $A_{j_1} \dots , A_{j_k}$ donde $j_1 < \dots < j_k < i$, usando una \textbf{regla de inferencia}.
    \end{block} \pause
    
    \begin{itemize}[<+->]
        \item Para $A_n$, la última de las fórmulas en la sequencia $S$, decimos que $A_n$ es un \alert{teorema}.
        \item La secuencia $S$ es una \textbf{prueba} de $A_n$.
        \item $A_n$ es \textit{demostrable}, denotado como $\vdash A_n$
        \item Si $\vdash A$, entonces $A$ puede usarse como \textbf{axioma} en una prueba subsecuente.
    \end{itemize}

\end{frame}

\begin{frame}{El proceso de inferencia}{Inferencia}

    Usando ciertas \textbf{fórmulas} de nuestro sistema de deducción, podemos llegar a alguna \textbf{conclusión}. \pause

    \bigskip
    
    Estas fórmulas que usamos para llegar a la conclusión son conocidas como \alert{premisas}, y a la \textit{verdad} a la que llegamos se le llama conclusión.

    \bigskip

    \begin{center}
        \footnotesize Podemos decir prácticamente que lo que concluimos es la conclusión, pero suena un poco redundante\dots    
    \end{center}
\end{frame}

\section{Reglas de Inferencia}

\begin{frame}{Modus Ponens}{Reglas de Inferencia}
    \begin{exampleblock}{Ejemplo}
        Díjole Esther a Edith: Si comes tus verduras, entonces puedes comerte una galleta de postre.
        Acto seguido, Edith se comió sus verduras sin pensarlo dos veces.

        \begin{itemize}
            \item Sabemos que \textbf{Si} Edith se come sus verduras, \textbf{entonces} puede comer una galleta de postre
            \item Sabemos que Edith se comió sus verduras.
        \end{itemize}

        Podemos entonces \textbf{deducir} que Edith puede comerse una galleta de postre.
    \end{exampleblock}

    \begin{block}{Definición de Modus Ponens}
            $$\inference[MP:]{P \implies Q \\ P}{Q}$$
    \end{block}
\end{frame}

\begin{frame}{Modus Tollens}{Reglas de Inferencia}

    \begin{exampleblock}{Mismo ejemplo}
        \begin{itemize}
            \item Sabemos que \textbf{Si} Edith se come sus verduras, \textbf{entonces} puede comer una galleta de postre
            \item Sabemos que Edith NO tuvo autorización de comerse una galleta de postre
        \end{itemize}

        Podemos entonces deducir que Edith no se comió sus verduras.
    \end{exampleblock}

    \begin{block}{Definición de Modus Tollens}
        $$\inference{P \implies Q \\ \neg Q}{\neg P}$$
    \end{block}
\end{frame}

\begin{frame}{Resolución}{Reglas de inferencia}

    \begin{exampleblock}{Ejemplo}
        \footnotesize
        Sospecho que Pedro o Gabriel se comieron el Gansito que dejé en el refri.
        Sin embargo, Pedro estaba en CDMX cuando el Gansito desapareció.
        Alfredo abrió el refri un par de veces el día que desapareció el Gansito.

        \begin{itemize}
            \item Sabemos que Pedro o Gabriel pudieron haberse comido el Gansito.
            \item Pedro no fue.
            \item Alfredo pudo haber sido.
            \item Pedro no fue o bien fue Alfredo.
        \end{itemize}

        Puedo deducir entonces que Gabriel o Alfredo pudieron haberse comido el Gansito.
    \end{exampleblock}

    \begin{block}{Definición de Resolución}
        $$\inference[Re:]{P \vee  Q \\  \neg P \vee R}{Q \vee R}$$
    \end{block}
\end{frame}

\section{Haciendo nuevas reglas}

\begin{frame}{Recogiendo Fresas}{Haciendo nuevas Reglas}

    Lo que sabemos:

    \begin{center}
        \textit{
        Si está soleado y cálido, entonces disfruto.
        Si está cálido y agradable, entonces recojo fresas.
        Si está lloviendo, no recojo fresas.
        Si está lloviendo, entonces me mojo.
        Está cálido.
        Está lloviendo.
        Está soleado.}
    \end{center} \pause
    \bigskip
    Lo que queremos probar:
    \begin{itemize}
        \item \textit{No voy a recoger fresas}
        \item \textit{Voy a disfrutar}
        \item \textit{Me voy a mojar}
    \end{itemize}
     \pause
    \bigskip

    ¿Qué proceso debemos seguir?
\end{frame}

\begin{frame}{Recogiendo Fresas II}{Haciendo nuevas reglas}

    \begin{enumerate}
        \item Establecer vocabulario
        \item Convertir todo a \texttt{OR}s y \texttt{AND}s (\alert{Forma Normal Conjuntiva})
        \item Negar la conclusión e incluirla en la base de conocimiento
        \item Usando resolución, intentar llegar a una fórmula vacía.
    \end{enumerate}
    
\end{frame}

\section{Introducción a la Prueba Matemática}

\begin{frame}{Algunos métodos de prueba}{Introducción a la prueba matemática}

    \begin{itemize}
        \itemsep2.5ex
        \item Contradicción: lo contrario a lo que quiero probar me lleva a una fórmula no factible.
        \item Tablas de verdad: si dos columnas son iguales, son equivalentes.
        \item Inferencia/Deducción: si llego al final usando reglas de inferencia válidas.
        \item Sustitución: si llego de una fórmula a otra equivalente usando equivalencias.
    \end{itemize}

\end{frame}


% \section*{Referencias}

% \begin{frame}[t]{Referencias}
    % \nocite{bibID01}
    % \nocite{bibID02}

    % \bibliographystyle{IEEE}
    % \bibliography{biblio}
% \end{frame}

\end{document}