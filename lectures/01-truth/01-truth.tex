\documentclass[spanish, c]{beamer}

\usepackage[utf8]{inputenc}
%\usepackage[spanish, mexico]{babel}
\usepackage{amsmath}
\usepackage{mathtools}
\usepackage{hyperref}
\usepackage{xcolor}
\usepackage{color}
\usepackage{ragged2e}
\usepackage{mathrsfs}
\usepackage{csquotes}
\usepackage{listings}
\usepackage[scaled]{beramono}
\usepackage[T1]{fontenc}
\usepackage{matlab-prettifier}
\usepackage{graphicx}
\usepackage{booktabs}

\renewcommand{\indent}{\hspace*{2em}}

% \usepackage{tikz}

% \usetikzlibrary{fit, shapes, arrows}

% \usepackage{courier}
% \usepackage{subfigure}
% \usepackage{enumerate}
% \usepackage{algorithmic}
% \usepackage{algorithm}

% \usepackage{listings}
% \usepackage{lstlinebgrd}

\usetheme{Boadilla}
\usefonttheme[onlymath]{serif}

\newcommand{\matlab}[1]{\lstinline[style=Matlab-editor]!#1!}
\newcommand\blfootnote[1]{%
\begingroup
\renewcommand\thefootnote{}\footnote{#1}%
\addtocounter{footnote}{-1}%
\endgroup
}

\lstset
{
    language = Matlab,
    style = Matlab-editor,
    basicstyle = \mlttfamily\scriptsize,
    escapechar = `,
    numbers = left,
    frame = tb,
}

\lstdefinestyle{output}
{
    language = {},
    basicstyle = \mlttfamily\scriptsize,
    escapechar = `,
    numbers = none,
    showtabs = false,
   	showstringspaces = false,
}

% Sets the templates
\definecolor{navyblue}{RGB}{0, 0, 128}
\definecolor{crimson}{RGB}{128, 16, 0}

\setbeamertemplate{navigation symbols}{}
\setbeamertemplate{headline}{}
\setbeamertemplate{title page}[default][colsep=-4bp,rounded=true]
\setbeamertemplate{footline}[frame number]
\setbeamertemplate{bibliography item}[text]
\setbeamertemplate{theorems}[numbered]

\setbeamercolor{title}{fg=navyblue, bg=white}
\setbeamercolor{frametitle}{fg=navyblue, bg=white}
\setbeamercolor{structure}{fg=navyblue}
\setbeamercolor{button}{fg=white,bg=navyblue}

\setbeamercovered{transparent}

\title{El estudio de la verdad}
\subtitle{Matemáticas Discretas \\ (TC1003)}
\author{
    \texorpdfstring{
        \begin{center}
            M.C. Xavier Sánchez Díaz \\
            \href{mailto:mail@tec.mx}{\texttt{mail@tec.mx}}
        \end{center}
    }
    {M.C. Xavier Sánchez Díaz}
}

\institute[Tecnológico de Monterrey]{\includegraphics[scale=0.5]{../img/logo}}
\date{}

\begin{document}

\setlength{\rightskip}{0pt}

\begin{frame}[plain]
    \titlepage        
\end{frame}

\begin{frame}{Outline}
    \tableofcontents
\end{frame}

\section{Del habla a las matemáticas I}

\begin{frame}{La verdad}{Del habla a las matemáticas I}
    
    \begin{itemize}[<+->]
        \itemsep2.5ex
        \item ¿Qué es la verdad?
        \item ¿Es posible decir siempre la verdad?
        \item ¿Existe la \textit{verdad absoluta}?
    \end{itemize}

    \bigskip

    \visible<4>{%
        Para poder confiar en que estamos haciendo las cosas bien, es necesario siempre \textbf{actuar con la verdad}.
    }

\end{frame}

\begin{frame}{Declaraciones}{Del habla a las matemáticas I}

    ¿Cuál de los siguientes enunciados es una declaración? \pause

    \begin{itemize}[<+->]
        \itemsep1.5ex
        \item \alert<11->{La vida es bella}
        \item $x + 3 = 0$
        \item Tanto hoy, como ayer\dots
        \item \alert<12->{Los amorosos callan}
        \item \alert<13->{Quiero un poco de pastel}
        \item ¿Quieres un poco de pastel?
        \item Pásame la botella
        \item Toma que toma, diablo arrastrado
        \item \alert<14->{La clave LADA de Monterrey es 81}
    \end{itemize}

\end{frame}

\begin{frame}{Declaraciones de a de veras}{Del habla a las matemáticas I}
    
    Las declaraciones pueden tener un valor de verdad. ¿Es cierto algo de esto? \pause

    \bigskip

    \begin{itemize}
        \item La vida es bella
        \item Los amorosos callan
        \item Quiero un poco de pastel
        \item La clave LADA de Monterrey es 81
    \end{itemize} \pause

    \bigskip

    Cuando una declaración puede tener un valor de verdad, se le llama \alert{estatuto}.    

\end{frame}

\begin{frame}{Estatutos}{Del habla a las matemáticas I}

    ¿Cuáles de los siguientes son estatutos? \pause

    \begin{itemize}[<+->]
        \item El sol sale por el este
        \item Si terminan 10 \textit{problem sets} hoy, nos vamos temprano
        \item Ya que tú eres un hombre y yo soy una mujer, ¿Por qué no vienes conmigo a mi costosa cama?\blfootnote{Ratón Esponja, Episodio 3}
        \item Me bañaré rápido e iré a tu casa
        \item Si me conoces, sabes que te quiero. Si no, seguirás tu camino
        \item El pasto es más verde del otro lado de la cerca
        \item Esta \textit{slide} es muy extensa
    \end{itemize}

\end{frame}

\begin{frame}{Atomicidad de los estatutos}{Del habla a las matemáticas I}

    Un estatuto \alert{atómico} o \textbf{primitivo} es aquél que no puede ser descompuesto en estatutos más pequeños. \pause

    \bigskip

    \begin{itemize}[<+->]
        \item El pasto es más verde del otro lado de la cerca
        \item Esta slide es muy extensa
    \end{itemize}
    
    \visible<4->{Cuando una declaración se puede separar en estatutos más pequeños, se le conoce como \alert{estatuto molecular} (o \textbf{compuesto}).} \pause 

    \begin{itemize}[<+->]
        \item El sol sale por el este o el sol sale por el oeste {\tiny (Una de dos\dots)}
        \item Me bañaré rápido e iré a tu casa
        \item El pasto es más verde del otro lado de la cerca y esta slide es muy extensa
    \end{itemize}
\end{frame}

\section{Del habla a las matemáticas II}

\begin{frame}{Variables y conectivos}{Del habla a las matemáticas II}

    Para darle más formalidad al asunto, emplearemos \textbf{variables atómicas} para denotar \alert{átomos}. \pause

    \bigskip

    \begin{center}
        \Large
        $P =$ El sol sale por el este

        $Q =$ El sol sale por el oeste
    \end{center} \pause

    \bigskip

    ¿Cómo las unimos?
\end{frame}

\begin{frame}{Variables y conectivos}{Del habla a las matemáticas II}
    
    Para conectar estatutos atómicos, usamos \alert{conectivos}: \pause

    \bigskip

    \begin{description}[<+->]
        \itemsep3.5ex
        \item \alert{Conjunción}: \\ \textit{Te quiero con limón \textbf<5->{y} sal.}
        \item \alert{Disyunción}: \\ \textit{Su sola existencia me molesta. Es ella \textbf<6->{o} yo. Tú decides.}
        \item \alert{Implicación}: \\ \textit{\textbf<7->{Si} sacas 90 en el examen, \only<7->{(\textbf{entonces})} apenas y pasas con 70. Ponte al tiro.}
    \end{description}

\end{frame}

\begin{frame}{Operadores lógicos}{Del habla a las matemáticas II}

    Cada \textbf{conectivo} tiene un \textbf{símbolo} en matemáticas: \pause

    \bigskip

    \begin{itemize}
        \itemsep2.5ex
        \item Para la conjunción (o el \textbf{``y''}), usamos el símbolo $\wedge$
        \item Para la disyunción (o la \textbf{``o''}), usamos el símbolo $\vee$
        \item Para la implicación (o el ``\textbf{si} \dots \textbf{entonces}'' \dots), usamos el símbolo $\implies$
    \end{itemize}

\end{frame}

\begin{frame}{Operadores lógicos}{Del habla a las matemáticas II}

    Sean $P$ el estatuto \textit{Te quiero con limón} y $Q$ el estatuto \textit{Te quiero con sal}. ¿Cuál es la \textbf{conjunción} de $P$ con $Q$? \pause

    \bigskip

    \begin{exampleblock}{Conjunción}
            $P \wedge Q$, que significa ``Te quiero con limón y te quiero con sal''.
    \end{exampleblock} \pause

    \bigskip

    ¿Cómo son la disyunción y la implicación entre $P$ y $Q$? \pause ¿Es lo mismo $P \implies Q$ que $Q \implies P$?

\end{frame}

\begin{frame}{Operadores lógicos}{Del habla a las matemáticas II}
    Además de los conectivos que ya vimos, existen \textbf{otros operadores} útiles: \pause

    \bigskip

    \begin{description}[<+->]
        \item \alert{Negación}: \\ \textit{El sol no sale por el oeste}
        \item \alert{Doble implicación} \\ \textit{$2x$ es mayor que $100$ si y sólo si $x$ vale más de 50}
    \end{description}

    \bigskip

    \visible<4->{%
        $\neg P$ significa \textit{no es cierto que} $P$. ¿Qué es $P$?

        \bigskip

        $P \iff Q$ es la doble implicación entre P y Q. ¿Qué pasa con este operador?
    }
\end{frame}


% What is control flow
% why is it important
% does it exist in math?
% how to represent it
% how to represent it in matlab
% practical cases

% \section*{Referencias}

% \begin{frame}[t]{Referencias}
    % \nocite{bibID01}
    % \nocite{bibID02}

    % \bibliographystyle{IEEE}
    % \bibliography{biblio}
% \end{frame}

\end{document}