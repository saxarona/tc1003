\documentclass[spanish, c]{beamer}

\usepackage[utf8]{inputenc}
\usepackage[spanish, mexico]{babel}
\usepackage{amsmath}
\usepackage{mathtools}
\usepackage{hyperref}
\usepackage{xcolor}
\usepackage{color}
\usepackage{ragged2e}
\usepackage{mathrsfs}
\usepackage{csquotes}
\usepackage{listings}
\usepackage[scaled]{beramono}
\usepackage[T1]{fontenc}
\usepackage{matlab-prettifier}
\usepackage{graphicx}
\usepackage{booktabs}
\usepackage{tikz}
\usepackage{venndiagram}

\renewcommand{\indent}{\hspace*{2em}}

% \usetikzlibrary{fit, shapes, arrows}

% \usepackage{courier}
% \usepackage{subfigure}
% \usepackage{enumerate}
% \usepackage{algorithmic}
% \usepackage{algorithm}

% \usepackage{listings}
% \usepackage{lstlinebgrd}

\usetheme{Boadilla}
\usefonttheme[onlymath]{serif}

\newcommand{\matlab}[1]{\lstinline[style=Matlab-editor]!#1!}
\newcommand\blfootnote[1]{%
\begingroup
\renewcommand\thefootnote{}\footnote{#1}%
\addtocounter{footnote}{-1}%
\endgroup
}

\lstset
{
    language = Matlab,
    style = Matlab-editor,
    basicstyle = \mlttfamily\scriptsize,
    escapechar = `,
    numbers = left,
    frame = tb,
}

\lstdefinestyle{output}
{
    language = {},
    basicstyle = \mlttfamily\scriptsize,
    escapechar = `,
    numbers = none,
    showtabs = false,
   	showstringspaces = false,
}

% Sets the templates
\definecolor{navyblue}{RGB}{0, 0, 128}
\definecolor{crimson}{RGB}{128, 16, 0}

\setbeamertemplate{navigation symbols}{}
\setbeamertemplate{headline}{}
\setbeamertemplate{title page}[default][colsep=-4bp,rounded=true]
\setbeamertemplate{footline}[frame number]
\setbeamertemplate{bibliography item}[text]
\setbeamertemplate{theorems}[numbered]

\setbeamercolor{title}{fg=navyblue, bg=white}
\setbeamercolor{frametitle}{fg=navyblue, bg=white}
\setbeamercolor{structure}{fg=navyblue}
\setbeamercolor{button}{fg=white,bg=navyblue}

\setbeamercovered{transparent}

\title{Conjuntos, colecciones y enumeración}
\subtitle{Matemáticas Discretas \\ (TC1003)}
\author{
    \texorpdfstring{
        \begin{center}
            M.C. Xavier Sánchez Díaz \\
            \href{mailto:sax@tec.mx}{\texttt{sax@tec.mx}}
        \end{center}
    }
    {M.C. Xavier Sánchez Díaz}
}

\institute[Tecnológico de Monterrey]{\includegraphics[scale=0.5]{../img/logo}}
\date{}

\begin{document}

\setlength{\rightskip}{0pt}

\begin{frame}[plain]
    \titlepage        
\end{frame}

\begin{frame}{Outline}
    \tableofcontents
\end{frame}

\section{Definición y propiedades}

\begin{frame}{¿Qué es un conjunto?}{Definición y propiedades de los conjuntos}

    Un \alert{conjunto} es un concepto abstracto, construido para referirse a una \textbf{colección} de \textbf{elementos}. \pause

    \bigskip

    Usualmente representamos los \alert{conjuntos} con letras mayúsculas (usualmente usando letras próximas a la $A$), y delimitamos sus contenidos con llaves (\textit{curly brackets}):

    \bigskip

    \begin{exampleblock}{Ejemplo}
        $A$ es el conjunto de los primeros cinco \textbf{números naturales}, es decir aquellos que \textit{nos sirven para contar}:
        $$A = \{1, 2, 3, 4, 5\}$$    
    \end{exampleblock}
\end{frame}

\begin{frame}{¿Qué es un conjunto?}{Definición y propiedades de los conjuntos}
    Un \textbf{conjunto} puede \alert{enumerarse} o \alert{describirse}:

    \bigskip

    \begin{exampleblock}{Enumeración}
        $$A = \{1, 2, 3, 4, 5 \}$$
    \end{exampleblock}

    \bigskip

    \begin{exampleblock}{Descripción}
        \begin{itemize}
            \item $A = $ el conjunto de los primeros cinco números naturales
            \item $B = $ el conjunto de personas en este salón
            \item $C = $ el conjunto de estudiantes del Campus Monterrey
        \end{itemize}
    \end{exampleblock}
\end{frame}

\begin{frame}{¿Qué es un conjunto?}{Definición y propiedades de los conjuntos}
    Un \textbf{conjunto} es una colección en la que \alert{no existe orden alguno}:
    \bigskip
    \begin{exampleblock}{Ejemplo}
        Si $A = \{1, 2, 3, 4, 5\}$ y $B = \{2, 3, 1, 5, 4\}$\dots

        \begin{itemize}
            \item ¿Cuál de los dos es el conjunto de los cinco primeros números naturales?
            \item ¿Cuáles son los elementos del primer conjunto y cuáles son los del segundo?
        \end{itemize}
    \end{exampleblock}

    Podemos usar el símbolo $\in$ para denotar \textit{pertenencia}, e.g. $2 \in A$ significa que el 2 es un elemento \textit{que pertenece} a $A$ o \textit{que está} en $A$.
\end{frame}

\begin{frame}{¿Qué es un conjunto?}{Definición y propiedades de los conjuntos}
    
    Podemos \textbf{contar los elementos} que hay dentro de un conjunto.
    A la \textbf{cantidad de elementos} dentro de un conjunto le llamamos \alert{cardinalidad}. \pause

    \bigskip

    \begin{exampleblock}{Ejemplo}
        Si $A = \{1, 2, 3, 4, 5\}$\dots

        \begin{itemize}
            \item \textbf{Q}: ¿Cuál es la cardinalidad de $A$?
            \item \textbf{A}: 5
        \end{itemize}
    \end{exampleblock} \pause

    \begin{block}{Nota}
        Aunque es poco común, a veces pueden observarse conjuntos con elementos repetidos.
        Si este fuera el caso, asume que sólo existe una copia de cada elemento.
    \end{block}

\end{frame}

\begin{frame}{¿Qué es un conjunto?}{Definición y propiedades de los conjuntos}    
    \begin{center}
        \Huge
        Story time: \textit{conjuntos finitos e infinitos}
    \end{center}
\end{frame}

\begin{frame}{¿Qué es un conjunto?}{Definición y propiedades de los conjuntos}
    Para denotar la \textbf{cardinalidad} de un conjunto \textit{contable}, usualmente usamos el símbolo $\#(A)$, mientras que usamos dos barras verticales para denotar la cardinalidad de un conjunto no contable.

    \begin{exampleblock}{Ejemplo}
        Si $A = \{1, 2, 3, 4, 5\}$ entonces $\#(A)= 5$ o bien $\vert A \vert = 5$
    \end{exampleblock} \pause

    Algunos autores usan una notación; otros, otra. No importa cuál usemos, intentemos ser consistentes.
\end{frame}

\section{Operaciones con conjuntos}

\begin{frame}{Inclusión}{Operaciones con conjuntos}
    Podemos \textbf{comparar} dos conjuntos en cuanto a tamaño, pero también podemos saber si uno está \alert{incluido} dentro de otro. \pause

    \begin{exampleblock}{Ejemplo}
        Si $A = $ el conjunto de habitantes de Nuevo León y $B = $ es el conjunto de habitantes de Monterrey, entonces sabemos que $B$ es un \alert{subconjunto} de $A$.
    \end{exampleblock} \pause

    Usamos la notación $B \subseteq A$ para decir que $B$ es un \alert{subconjunto} de $A$; \textit{cada elemento de $B$ está en $A$}\dots \pause

    \bigskip

    {\Large PERO ESPERA}
\end{frame}

\begin{frame}{Inclusión}{Operaciones con conjuntos}

    Si $A = $ el conjunto de habitantes de Nuevo León y $B = $ es el conjunto de habitantes de Monterrey, entonces sabemos que $B$ es un \alert{subconjunto \textbf{propio}} de $A$: \pause

    \begin{block}{Inclusión propia}
        \begin{itemize}
            \item Si \textbf{todos} los elementos de $B$ están en $A$, sabemos que $B \subseteq A$.
            \item Si \textbf{todos} los elementos de $B$ están en $A$, pero no todos los elementos de $A$ están en $B$, entonces $B \subset A$
        \end{itemize}        
    \end{block}

    A esto último se le llama inclusión propia (que es el caso de los de Monterrey y los de Nuevo León), y da \textit{más información} que la simple inclusión.

    \bigskip
    
    {\Large Mini-story time: \textit{orden estricto}}

\end{frame}

\begin{frame}{Identidad}{Operaciones con conjuntos}

    ¿Qué información tenemos en cada caso? Reflexiona un momento\dots \pause

    \bigskip

    \begin{itemize}[<+->]
        \item $A \subseteq B$
        \item $B \subseteq A$
        \item $A \subseteq B$ y $B \subseteq A$
    \end{itemize} \pause

    Cuando dos conjuntos tienen lo mismo, decimos que son \alert{idénticos} {\tiny (duh)}.

    \bigskip \pause

    Observa la diferencia entre $A \subseteq B$ y $A \subset B$.
\end{frame}

\begin{frame}{Más sobre subconjuntos}{Operaciones con conjuntos}
    \begin{center}
        {\LARGE
        ¿Cuántos subconjuntos posibles puedes enumerar para el conjunto $A = \{2, 4, 6, 8, 10\}$?}
    
        \bigskip

        \textit{Hint: considera las distintas maneras de meter sus elementos a una sub-caja.}
    \end{center}
\end{frame}

\begin{frame}{El Conjunto Potencia}{Operaciones con conjuntos}
    El \alert{conjunto potencia}  de $A$, denotado por $\wp(A)$, es el \textbf{conjunto de todos los posibles subconjuntos} en $A$. \pause

    \bigskip

    \begin{theorem}
        $\vert \wp(A) \vert = 2^{\vert A \vert}$.
    \end{theorem} \pause

    \bigskip

    Puedes enumerar todos los elementos de $\wp(A)$ si $A = \{1, 2, 3, 4\}$?
\end{frame}

\begin{frame}{El conjunto vacío}{Operaciones con conjuntos}
    Cualquier conjunto $A$ el cual no tiene elementos, i.e. $\vert A \vert = 0$ es un \alert{conjunto vacío}. \pause

    \bigskip

    El conjunto vacío, usualmente representado como $\emptyset$ o $\{\}$ es una parte \textbf{esencial} de la teoría de conjuntos: \pause

    \begin{itemize}[<+->]
        \item Es siempre un subconjunto de \textbf{cualquier conjunto}, i.e., $\emptyset \subset A$
        \item Es equiparable a \textit{la nada}, que es distinto del 0.
    \end{itemize} \pause

    \bigskip

    {\Large Mini story time: \textit{null vs 0}}

\end{frame}

\begin{frame}{La lógica en los conjuntos}{Operaciones con conjuntos}
    Podemos conectar el tema de \textbf{lógica proposicional} con los conjuntos de muchas maneras.
    La primera de ella es para \alert{describir} los conjuntos. \pause

    \begin{exampleblock}{Ejemplo}
        El conjunto $A$ de los primeros 5 números naturales, podemos describirlo \textit{formalmente} como
        $$A = \{n \, \mid \, n \in \mathbb{N} \wedge n \leq 5 \}$$
    \end{exampleblock} \pause

    \begin{block}{Nota}
        Algunos autores usan $\mid$ y otros $:$.
        Cualquiera de los dos jala. Luego veremos aplicaciones extras de cada símbolo.
    \end{block}
    \bigskip
    {\Large Mini story time: \textit{$\mathbb{N}$} y uso de , y ;}
\end{frame}

\begin{frame}{La lógica en los conjuntos: Intersección}{Operaciones con conjuntos}

    De igual manera, existe una \textbf{correspondencia} de cada uno de los operadores lógicos que vimos con \textbf{operaciones con conjuntos}: \pause

    \bigskip

    \begin{block}{Intersección y Conjunción}
        $$x \in A \wedge x \in B \vdash x \in A \cap B$$
    \end{block} \pause

    \bigskip

    ¿Qué significa esto? \pause

    \bigskip

    La intersección, así como la conjunción, es asociativa, conmutativa y distributiva sobre la unión.

\end{frame}

\begin{frame}{Diagramas de Venn}{Operaciones con conjuntos}

    Un diagrama de Venn es una manera efectiva de ilustrar las relaciones entre dos conjuntos. \pause

    \begin{center}
        \begin{venndiagram2sets}
            \fillACapB
        \end{venndiagram2sets}
    \end{center}

\end{frame}

\begin{frame}{La lógica en los conjuntos: Unión}{Operaciones con conjuntos}

    \begin{block}{Unión y Disyunción}
        $$x \in A \vee x \in B \vdash x \in A \cup B$$
    \end{block}

    \bigskip

    La unión, así como la disyunción, es asociativa, conmutativa y distributiva sobre la intersección.

\end{frame}

\begin{frame}{La lógica en los conjuntos: Complemento}{Operaciones con conjuntos}
    \begin{block}{Complemento y Negación}
        $$x \notin A \vdash x \in \complement(A)$$
    \end{block}
    
    \bigskip

    Para poder contextualizar el \alert{complemento} de $A$ como \textit{todo aquello que no es } $A$, hay que delimitar qué es \textit{todo}. A ese \textit{todo} le llamamos \alert{universo}, y es usualmente representado con la letra $U$.

    \bigskip

    {\Large Mini story time: \textit{Proofwiki Complement definition\blfootnote{\url{https://proofwiki.org/wiki/Definition:Set\_Complement}}}}
\end{frame}

\begin{frame}{Operaciones adicionales: Diferencia}{Operaciones con conjuntos}
    La idea de tener un \textbf{universo} nos deja pensando en que el \textbf{complemento} de un conjunto $A$ es la \alert{diferencia} del \textbf{universo menos} $A$:

    \begin{block}{Diferencia}
        $$\complement(A) \equiv U - A \equiv U \setminus A$$
    \end{block}

    \bigskip

    \begin{block}{Ejemplo: diferencia}
        Si el universo consta de dos conjuntos $A$ y $B$, entonces
        $$x \in A \wedge x \not \in B \vdash x \in A \setminus B$$
    \end{block}
\end{frame}

\begin{frame}{Operaciones adicionales: Diferencia simétrica}{Operaciones con conjuntos}
    La \alert{diferencia simétrica} es una \textit{diferencia mutuamente excluyente}:

    \bigskip

    \begin{block}{Diferencia simétrica}
        Si el universo consta de dos conjuntos $A$ y $B$, entonces
        $$(x \in A \wedge x \not \in B) \vee x (\in B \wedge x \not \in A) \vdash x \in A \oplus B$$
    \end{block} \pause

    \bigskip

    ¿Cuál sería el diagrama de Venn para esta operación?
\end{frame}

\section{Equivalencias y leyes de conjuntos}

\begin{frame}{Equivalencias}{Equivalencias y leyes de conjuntos}

    \begin{itemize}
        \item $A \cup A \equiv A$
        \item $A \cap A \equiv A$
        \item $A \cup B \equiv B \cup A$
        \item $A \cap B \equiv B \cap A$
        \item $A \cup (B \cup C) \equiv (A \cup B) \cup C$
        \item $A \cap (B \cap C) \equiv (A \cap B) \cap C$
        \item $\complement(A \cup B) \equiv \complement(A) \cap \complement(B)$
        \item $\complement(A \cap B) \equiv \complement(A) \cup \complement(B)$
        \item $A \cap (A \cup B) \equiv A \equiv A \cup (A \cap B)$
        \item $A \cap (B \cup C) \equiv (A \cup B) \cap (A \cup C)$
        \item $A \cup (B \cap C) \equiv (A \cap B) \cup (A \cap C)$
        \item $\complement(A) \cap A \equiv \emptyset$
        \item $\complement(\complement(A)) = A$
        \item $A \cup \emptyset \equiv A$
        \item $A \cap \emptyset \equiv \emptyset$
    \end{itemize}
\end{frame}

% What is a set
% Why are sets important
% Example of sets
% Sets operations
%% Inclusion
%% Identity and proper inclusion
% Cardinality
% Venn Diagrams
% The Empty Set
% Disjoint Sets
% Boolean operations on Sets
%% Intersection vs Logical Conjuction
%% Union vs Logical Disjunction
%% Set Difference
%% Extra: Symmetric Difference
%% Set Complement
% Power Set
% Examples on set operations
% Describing a set
% 


% \section*{Referencias}

% \begin{frame}[t]{Referencias}
    % \nocite{bibID01}
    % \nocite{bibID02}

    % \bibliographystyle{IEEE}
    % \bibliography{biblio}
% \end{frame}

\end{document}