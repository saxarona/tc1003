\documentclass[]{book}

%These tell TeX which packages to use.
\usepackage{array,epsfig}
\usepackage{amsmath}
\usepackage{amsfonts}
\usepackage{amssymb}
\usepackage{amsxtra}
\usepackage{amsthm}
\usepackage{mathrsfs}
\usepackage{color}
\usepackage[spanish, mexico]{babel}
\usepackage[utf8]{inputenc}

%Here I define some theorem styles and shortcut commands for symbols I use often
\theoremstyle{definition}
\newtheorem{defn}{Definition}
\newtheorem{thm}{Theorem}
\newtheorem{cor}{Corollary}
\newtheorem*{rmk}{Remark}
\newtheorem{lem}{Lemma}
\newtheorem*{joke}{Joke}
\newtheorem{ex}{Example}
\newtheorem*{sol}{Solution}
\newtheorem{prop}{Proposition}

\newcommand{\lra}{\longrightarrow}
\newcommand{\ra}{\rightarrow}
\newcommand{\surj}{\twoheadrightarrow}
\newcommand{\graph}{\mathrm{graph}}
\newcommand{\bb}[1]{\mathbb{#1}}
\newcommand{\Z}{\bb{Z}}
\newcommand{\Q}{\bb{Q}}
\newcommand{\R}{\bb{R}}
\newcommand{\C}{\bb{C}}
\newcommand{\N}{\bb{N}}
\newcommand{\M}{\mathbf{M}}
\newcommand{\m}{\mathbf{m}}
\newcommand{\MM}{\mathscr{M}}
\newcommand{\HH}{\mathscr{H}}
\newcommand{\Om}{\Omega}
\newcommand{\Ho}{\in\HH(\Om)}
\newcommand{\bd}{\partial}
\newcommand{\del}{\partial}
\newcommand{\bardel}{\overline\partial}
\newcommand{\textdf}[1]{\textbf{\textsf{#1}}\index{#1}}
\newcommand{\img}{\mathrm{img}}
\newcommand{\ip}[2]{\left\langle{#1},{#2}\right\rangle}
\newcommand{\inter}[1]{\mathrm{int}{#1}}
\newcommand{\exter}[1]{\mathrm{ext}{#1}}
\newcommand{\cl}[1]{\mathrm{cl}{#1}}
\newcommand{\ds}{\displaystyle}
\newcommand{\vol}{\mathrm{vol}}
\newcommand{\cnt}{\mathrm{ct}}
\newcommand{\osc}{\mathrm{osc}}
\newcommand{\LL}{\mathbf{L}}
\newcommand{\UU}{\mathbf{U}}
\newcommand{\support}{\mathrm{support}}
\newcommand{\AND}{\;\wedge\;}
\newcommand{\OR}{\;\vee\;}
\newcommand{\Oset}{\varnothing}
\newcommand{\st}{\ni}
\newcommand{\wh}{\widehat}

%Pagination stuff.
\setlength{\topmargin}{-.3 in}
\setlength{\oddsidemargin}{0in}
\setlength{\evensidemargin}{0in}
\setlength{\textheight}{9.in}
\setlength{\textwidth}{6.5in}
\setlength{\itemsep}{0.45in}
\pagestyle{empty}



\begin{document}

\begin{center}
{\huge Matemáticas Discretas TC1003}\\[1.5ex]
{\large \textbf{Tarea 01}\\[1.5ex] %You should put your name here
23.02.20} %You should write the date here.
\end{center}

\vspace{0.2 cm}

\subsection*{Lógica proposicional}

Una empresa embotelladora tiene una base de datos con cuatro tablas distintas:
\textdf{Ventas}, \textdf{Auditoría}, \textdf{Censo}, y \textdf{Entorno}.
A su vez, los clientes de la empresa pertenecen a distintas áreas del sector comercial:
\textdf{Industrias y Empresas}, \textdf{Restaurantes}, \textdf{Tiendas de Conveniencia} y \textdf{Casa habitación}.

El equipo de análisis de datos aún no está seguro sobre cuál es el lenguaje que usará.
Mientras se deciden, te han pedido que dejes las consultas hechas a nivel matemático; luego las adaptarán al lenguaje que decidan emplear.

\bigskip

Comienza \textdf{generando el alfabeto lógico que utilizarás}\dots ¿qué variables usarás para denotar la \textit{pertenencia} de un cliente a cada una de las categorías anteriores? Cuando tengas tu alfabeto, escribe en simbología lógica las siguientes consultas:

\begin{enumerate}
	\item El cliente tiene ventas o bien datos demográficos de entorno
	\item El cliente es un restaurante y está en auditoría y censo pero no en entorno
	\item El cliente es una tienda de conveniencia o una casa habitación y no tiene auditoría
	\item El cliente es de cualquier tipo excepto casa habitación
	\item El cliente es una industria, y tenemos todos sus datos
	\item El cliente es una casa habitación y no tiene censo pero sí tiene auditoría
	\item El cliente es una empresa que tiene auditoría o censo, pero sólo uno de los dos
	\item El cliente es un restaurante del que tenemos sus ventas y, o bien censo y auditoría, o bien censo y entorno
	\item El cliente es un restaurante que no tiene censo ni entorno pero sí tiene ventas
	\item El cliente está mal etiquetado porque sale como tienda de conveniencia y como restaurante
\end{enumerate}

\bigskip

¿Puedes reducir alguna de ellas?

\bigskip

Entrega en un documento \textdf{PDF} y súbelo a Canvas, en el apartado adecuado.

\end{document}


