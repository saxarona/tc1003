\documentclass[]{book}

%These tell TeX which packages to use.
\usepackage{array,epsfig}
\usepackage{amsmath}
\usepackage{amsfonts}
\usepackage{amssymb}
\usepackage{amsxtra}
\usepackage{amsthm}
\usepackage{mathrsfs}
\usepackage{color}
\usepackage[spanish, mexico]{babel}
\usepackage[utf8]{inputenc}

%Here I define some theorem styles and shortcut commands for symbols I use often
\theoremstyle{definition}
\newtheorem{defn}{Definition}
\newtheorem{thm}{Theorem}
\newtheorem{cor}{Corollary}
\newtheorem*{rmk}{Remark}
\newtheorem{lem}{Lemma}
\newtheorem*{joke}{Joke}
\newtheorem{ex}{Example}
\newtheorem*{sol}{Solution}
\newtheorem{prop}{Proposition}

\newcommand{\lra}{\longrightarrow}
\newcommand{\ra}{\rightarrow}
\newcommand{\surj}{\twoheadrightarrow}
\newcommand{\graph}{\mathrm{graph}}
\newcommand{\bb}[1]{\mathbb{#1}}
\newcommand{\Z}{\bb{Z}}
\newcommand{\Q}{\bb{Q}}
\newcommand{\R}{\bb{R}}
\newcommand{\C}{\bb{C}}
\newcommand{\N}{\bb{N}}
\newcommand{\M}{\mathbf{M}}
\newcommand{\m}{\mathbf{m}}
\newcommand{\MM}{\mathscr{M}}
\newcommand{\HH}{\mathscr{H}}
\newcommand{\Om}{\Omega}
\newcommand{\Ho}{\in\HH(\Om)}
\newcommand{\bd}{\partial}
\newcommand{\del}{\partial}
\newcommand{\bardel}{\overline\partial}
\newcommand{\textdf}[1]{\textbf{\textsf{#1}}\index{#1}}
\newcommand{\img}{\mathrm{img}}
\newcommand{\ip}[2]{\left\langle{#1},{#2}\right\rangle}
\newcommand{\inter}[1]{\mathrm{int}{#1}}
\newcommand{\exter}[1]{\mathrm{ext}{#1}}
\newcommand{\cl}[1]{\mathrm{cl}{#1}}
\newcommand{\ds}{\displaystyle}
\newcommand{\vol}{\mathrm{vol}}
\newcommand{\cnt}{\mathrm{ct}}
\newcommand{\osc}{\mathrm{osc}}
\newcommand{\LL}{\mathbf{L}}
\newcommand{\UU}{\mathbf{U}}
\newcommand{\support}{\mathrm{support}}
\newcommand{\AND}{\;\wedge\;}
\newcommand{\OR}{\;\vee\;}
\newcommand{\Oset}{\varnothing}
\newcommand{\st}{\ni}
\newcommand{\wh}{\widehat}

%Pagination stuff.
\setlength{\topmargin}{-.3 in}
\setlength{\oddsidemargin}{0in}
\setlength{\evensidemargin}{0in}
\setlength{\textheight}{9.in}
\setlength{\textwidth}{6.5in}
\setlength{\itemsep}{0.45in}
\pagestyle{empty}



\begin{document}

\begin{center}
{\huge Matemáticas Discretas TC1003}\\[1.5ex]
{\large \textbf{Tarea 03}\\[1.5ex] %You should put your name here
26.04.20} %You should write the date here.
\end{center}

\vspace{0.2 cm}

\subsection*{Lógica de Primer Orden}

{\footnotesize \textit{El siguiente problema fue originalmente propuesto por el Dr. Santiago Conant-Pablos, profesor investigador del Grupo de Investigación con Enfoque Estratégico en Sistemas Inteligentes.}}

\bigskip

Demuestra o refuta mediante \textdf{reglas de inferencia} (que puede incluir resolución por refutación) que \textit{Plata no es un cerdo}, que \textit{los caballos no tienen alas} y que \textit{Plata es un caballo con alas que sabe silbar} si se sabe que:

\begin{itemize}
	\item \textit{Ningún caballo sabe silbar}
	\item \textit{Ningún cerdo tiene alas}
	\item \textit{Todos los que no saben silbar tienen alas}
	\item \textit{Plata es un caballo}
\end{itemize}

\bigskip

Para traducir a lógica de predicados, utiliza el siguiente vocabulario:

\begin{itemize}
	\item $Caballo(x) \longrightarrow$ $x$ es un caballo
	\item $Silbar(x)\longrightarrow$ $x$ sabe silbar
	\item $Alas(x) \longrightarrow$ $x$ tiene alas
	\item $Cerdo(x)\longrightarrow$ $x$ es un cerdo
\end{itemize}

\bigskip

Entrega en un documento \textdf{PDF} en \textdf{typesetting} (es decir, formato nativamente digital, \textbf{no escaneo}) y súbelo a Canvas, en el apartado adecuado.

\end{document}


