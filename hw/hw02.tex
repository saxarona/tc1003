\documentclass[]{book}

%These tell TeX which packages to use.
\usepackage{array,epsfig}
\usepackage{amsmath}
\usepackage{amsfonts}
\usepackage{amssymb}
\usepackage{amsxtra}
\usepackage{amsthm}
\usepackage{mathrsfs}
\usepackage{color}
\usepackage[spanish, mexico]{babel}
\usepackage[utf8]{inputenc}

%Here I define some theorem styles and shortcut commands for symbols I use often
\theoremstyle{definition}
\newtheorem{defn}{Definition}
\newtheorem{thm}{Theorem}
\newtheorem{cor}{Corollary}
\newtheorem*{rmk}{Remark}
\newtheorem{lem}{Lemma}
\newtheorem*{joke}{Joke}
\newtheorem{ex}{Example}
\newtheorem*{sol}{Solution}
\newtheorem{prop}{Proposition}

\newcommand{\lra}{\longrightarrow}
\newcommand{\ra}{\rightarrow}
\newcommand{\surj}{\twoheadrightarrow}
\newcommand{\graph}{\mathrm{graph}}
\newcommand{\bb}[1]{\mathbb{#1}}
\newcommand{\Z}{\bb{Z}}
\newcommand{\Q}{\bb{Q}}
\newcommand{\R}{\bb{R}}
\newcommand{\C}{\bb{C}}
\newcommand{\N}{\bb{N}}
\newcommand{\M}{\mathbf{M}}
\newcommand{\m}{\mathbf{m}}
\newcommand{\MM}{\mathscr{M}}
\newcommand{\HH}{\mathscr{H}}
\newcommand{\Om}{\Omega}
\newcommand{\Ho}{\in\HH(\Om)}
\newcommand{\bd}{\partial}
\newcommand{\del}{\partial}
\newcommand{\bardel}{\overline\partial}
\newcommand{\textdf}[1]{\textbf{\textsf{#1}}\index{#1}}
\newcommand{\img}{\mathrm{img}}
\newcommand{\ip}[2]{\left\langle{#1},{#2}\right\rangle}
\newcommand{\inter}[1]{\mathrm{int}{#1}}
\newcommand{\exter}[1]{\mathrm{ext}{#1}}
\newcommand{\cl}[1]{\mathrm{cl}{#1}}
\newcommand{\ds}{\displaystyle}
\newcommand{\vol}{\mathrm{vol}}
\newcommand{\cnt}{\mathrm{ct}}
\newcommand{\osc}{\mathrm{osc}}
\newcommand{\LL}{\mathbf{L}}
\newcommand{\UU}{\mathbf{U}}
\newcommand{\support}{\mathrm{support}}
\newcommand{\AND}{\;\wedge\;}
\newcommand{\OR}{\;\vee\;}
\newcommand{\Oset}{\varnothing}
\newcommand{\st}{\ni}
\newcommand{\wh}{\widehat}

%Pagination stuff.
\setlength{\topmargin}{-.3 in}
\setlength{\oddsidemargin}{0in}
\setlength{\evensidemargin}{0in}
\setlength{\textheight}{9.in}
\setlength{\textwidth}{6.5in}
\setlength{\itemsep}{0.45in}
\pagestyle{empty}



\begin{document}

\begin{center}
{\huge Matemáticas Discretas TC1003}\\[1.5ex]
{\large \textbf{Tarea 01}\\[1.5ex] %You should put your name here
01.03.20} %You should write the date here.
\end{center}

\vspace{0.2 cm}

\subsection*{Conjuntos}

Siguiendo con la empresa embotelladora con las tablas de \textdf{Ventas}, \textdf{Auditoría}, \textdf{Censo}, y \textdf{Entorno};
y clientes en sectores \textdf{Industrias y Empresas}, \textdf{Restaurantes}, \textdf{Tiendas de Conveniencia} y \textdf{Casa habitación},
el equipo de análisis de datos quiere expresar el resultado de cada una de las consultas pasadas como un \textbf{subconjunto} de todos sus clientes.

\bigskip

Comienza \textdf{generando el alfabeto que usarás para tus conjuntos}\dots ¿qué variables usarás para denotar cada uno?

Cuando tengas tu alfabeto, escribe como un conjunto por \textdf{descripción} cada una de las siguientes consultas:

\begin{enumerate}
	\item El conjunto de clientes que tienen ventas o bien datos demográficos de entorno
	\item El conjunto de clientes que son restaurantes y están en auditoría y en censo pero no en entorno
	\item El conjunto de clientes que son tiendas de conveniencia o casa habitación, y no tienen auditoría
	\item El conjunto de clientes que son de cualquier tipo excepto casa habitación
	\item El conjunto de clientes que son industrias, y tenemos todos sus datos
	\item El conjunto de clientes que son casas habitación y no tienen censo pero sí tienen auditoría
	\item El conjunto de clientes que son empresas que tienen auditoría o censo, pero sólo uno de los dos
	\item El conjunto de clientes que son restaurantes de los que tenemos sus ventas y, o bien censo y auditoría, o bien censo y entorno
	\item El conjunto de clientes que son restaurantes que no tienen censo ni entorno pero sí tienen ventas
	\item El conjunto de clientes que están mal etiquetados porque salen como tiendas de conveniencia y como restaurantes
\end{enumerate}

\bigskip

¿Hay algunas de estas consultas que sean redundantes?

\bigskip

Entrega en un documento \textdf{PDF} y súbelo a Canvas, en el apartado adecuado.

\end{document}


