\documentclass[]{book}

%These tell TeX which packages to use.
\usepackage{array,epsfig}
\usepackage{amsmath}
\usepackage{amsfonts}
\usepackage{amssymb}
\usepackage{amsxtra}
\usepackage{amsthm}
\usepackage{mathrsfs}
\usepackage{color}
\usepackage[spanish, mexico]{babel}
\usepackage[utf8]{inputenc}
\usepackage{tcolorbox}

%Here I define some theorem styles and shortcut commands for symbols I use often
\theoremstyle{definition}
\newtheorem{defn}{Definition}
\newtheorem{thm}{Theorem}
\newtheorem{cor}{Corollary}
\newtheorem*{rmk}{Remark}
\newtheorem{lem}{Lemma}
\newtheorem*{joke}{Joke}
\newtheorem{ex}{Example}
\newtheorem*{sol}{Solution}
\newtheorem{prop}{Proposition}

\newcommand{\lra}{\longrightarrow}
\newcommand{\ra}{\rightarrow}
\newcommand{\surj}{\twoheadrightarrow}
\newcommand{\graph}{\mathrm{graph}}
\newcommand{\bb}[1]{\mathbb{#1}}
\newcommand{\Z}{\bb{Z}}
\newcommand{\Q}{\bb{Q}}
\newcommand{\R}{\bb{R}}
\newcommand{\C}{\bb{C}}
\newcommand{\N}{\bb{N}}
\newcommand{\M}{\mathbf{M}}
\newcommand{\m}{\mathbf{m}}
\newcommand{\MM}{\mathscr{M}}
\newcommand{\HH}{\mathscr{H}}
\newcommand{\Om}{\Omega}
\newcommand{\Ho}{\in\HH(\Om)}
\newcommand{\bd}{\partial}
\newcommand{\del}{\partial}
\newcommand{\bardel}{\overline\partial}
\newcommand{\textdf}[1]{\textbf{\textsf{#1}}\index{#1}}
\newcommand{\img}{\mathrm{img}}
\newcommand{\ip}[2]{\left\langle{#1},{#2}\right\rangle}
\newcommand{\inter}[1]{\mathrm{int}{#1}}
\newcommand{\exter}[1]{\mathrm{ext}{#1}}
\newcommand{\cl}[1]{\mathrm{cl}{#1}}
\newcommand{\ds}{\displaystyle}
\newcommand{\vol}{\mathrm{vol}}
\newcommand{\cnt}{\mathrm{ct}}
\newcommand{\osc}{\mathrm{osc}}
\newcommand{\LL}{\mathbf{L}}
\newcommand{\UU}{\mathbf{U}}
\newcommand{\support}{\mathrm{support}}
\newcommand{\AND}{\;\wedge\;}
\newcommand{\OR}{\;\vee\;}
\newcommand{\Oset}{\varnothing}
\newcommand{\st}{\ni}
\newcommand{\wh}{\widehat}

%Pagination stuff.
\setlength{\topmargin}{-.3 in}
\setlength{\oddsidemargin}{0in}
\setlength{\evensidemargin}{0in}
\setlength{\textheight}{9.in}
\setlength{\textwidth}{6.5in}
\setlength{\itemsep}{0.45in}
\pagestyle{empty}



\begin{document}

\begin{center}
{\huge Matemáticas Discretas TC1003}\\[1.5ex]
{\large \textbf{Tarea 04}\\[1.5ex] %You should put your name here
26.04.20} %You should write the date here.
\end{center}

\vspace{0.2 cm}

\section*{Funciones, Lógica y Grafos}

% {\footnotesize \textit{El siguiente problema fue originalmente propuesto por el Dr. Santiago Conant-Pablos, profesor investigador del Grupo de Investigación con Enfoque Estratégico en Sistemas Inteligentes.}}

\bigskip

{\large Considera el siguiente problema de asignación de asientos y contesta lo que se te pide.}

\bigskip

\begin{tcolorbox}
\slshape El Consejo de la Ciudad tendrá una cena de gala, pero algunos de los invitados necesitan cierta atención, así que tanto la recepción como la asignación de asientos tiene que ser sumamente cuidadosa.
Con tres mesas disponibles (una de 5 y dos de 6 asientos cada una), el consejo quiere evitar conflictos de intereses siguiendo las restricciones a continuación:

\begin{itemize}
	\item La familia del Faraón (Papá, Mamá y dos niños) no pueden estar cerca del Sacerdote. Los niños están intrigados por el Robot, así que les gustaría sentarse en la misma mesa.
	\item Demis, Vangelis y Mikis (los griegos) tienen que sentarse en la misma mesa.
	\item Wilson y \AA kerfeldt quieren sentarse con los griegos, pero no quieren estar en la misma mesa que Parsons.
	\item Parsons y Gilmour preferirían sentarse juntos.
	\item Dickinson, Harris, Dio y Summers---los metaleros---son \textit{easy-going}, así que no tienen ninguna restricción. Sólo quieren cerveza.
	\item El Sacerdote no quiere cerca a los metaleros ni al Robot.
\end{itemize}
\end{tcolorbox}

\bigskip

\begin{enumerate}
	\item Describe formalmente las restricciones como una relación $R$:
	\begin{itemize}
		\item ¿Cuál es el dominio de $R$?
		\item ¿Cuál es el rango de $R$?
	\end{itemize}
	\item Describe formalmente el problema como un grafo $G$:
	\begin{itemize}
		\item ¿Cuál es el conjunto de vértices $V$?
		\item ¿Cuál es el conjunto de ejes $E$?
		\item Representa gráficamente a $G$ ¿Qué relación existe entre $R$ y $G$?
		\item Describe \textsl{informalmente} el grafo y sus características
	\end{itemize}
	\item Propón una asignación de asientos válida---que respete todas las restricciones (implícitas y explícitas) del problema.
\end{enumerate}


Entrega en un documento \textdf{PDF} en \textdf{typesetting} (es decir, formato nativamente digital, \textbf{no escaneo}) y súbelo a Canvas, en el apartado adecuado.

\end{document}
