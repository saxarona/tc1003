\documentclass[spanish, 10pt]{article}

\usepackage[table, xcdraw]{xcolor}
\usepackage[utf8]{inputenc}
\usepackage[spanish, mexico]{babel}
\usepackage{helvet}
\usepackage{fullpage}
\usepackage{graphicx}
\usepackage{enumitem}
\usepackage{tikz}
\usepackage{ulem}
\usepackage{url}
\usepackage{hyperref}
\usepackage[margin = 3 cm]{geometry}
\usepackage{amsmath}
\usepackage{amsfonts}
\usepackage{tcolorbox}

\usepackage{matlab-prettifier}
\usepackage{multicol}

\usetikzlibrary{arrows, shapes, trees, calc, decorations.pathreplacing, shapes.misc, positioning, automata}

\setlength\parindent{0pt}

\renewcommand{\familydefault}{\sfdefault}
\newcommand{\responserule}{{\large\rule{14 cm}{0.3mm}}}
\newcommand{\shortresponserule}{{\large\rule{5 cm}{0.3mm}}}
\newcommand{\veryshortresponserule}{{\large\rule{3 cm}{0.3mm}}}
\newcommand{\matlab}[1]{\lstinline[style=Matlab-pyglike]!#1!}


% Specifications for listing package
% \lstset{	
%     basicstyle = \scriptsize\ttfamily,
%     keywordstyle = \color{blue}\ttfamily,
%     stringstyle = \color{red}\ttfamily,
%    	commentstyle = \color{gray}\ttfamily,
%    	tabsize = 3,
%    	breaklines = true,
%    	stepnumber = 1,
%    	showtabs = false,
%    	showstringspaces = false,
%    	frame = none
% }

% Commands for true/false questions
% ----------------------------------------------------------------
\newcommand{\question}[1]{%
	\noindent
	\begin{minipage}[t]{0.15\linewidth}
	\centering		
		\textbf{[\hspace{1 cm}]}
	\end{minipage}%
	\begin{minipage}[t]{0.85\linewidth}
		#1
	\end{minipage}
	\smallskip
}
% ----------------------------------------------------------------

\setlength\parindent{0pt}

\begin{document}

\begin{center}
	{\Large \textbf{Matemáticas Discretas (TC-1003)}}
	
	\bigskip
	{\large \textbf{AEX03 -- Grafos}}
\end{center}

\bigskip
{\large \textbf{Nombre}: \rule{13.7 cm}{0.4mm}}

% \bigskip
% {\large \textbf{Matrícula}: \rule{5 cm}{0.4mm}}

% \bigskip
% {\large \textbf{Name}: \rule{14 cm}{0.4mm}}

\bigskip
{\large \textbf{Matrícula}: \rule{5 cm}{0.4mm}} \hfill {\large \textbf{Fecha}: \today}

\bigskip

% {\footnotesize Nota: es probable que esta actividad nos asuste un poco al principio. Es perfectamente normal.
% En efecto, es de mayor dificultad a las que hemos visto anteriormente y probablemente haya dudas.
% Si hay algo que no entiendas, no te quedes sin preguntar.}

\section{Árboles y expresiones S}

Una \textit{S-expression} (expresión S en español, también conocida como \texttt{sexpr} o \texttt{sexp}) es una manera conveniente de expresar una lista anidada de datos.
El nombre viene de \textit{expresión simbólica} y podemos explicar su comportamiento con el siguiente ejemplo:

\begin{enumerate}
	\item Piensa en la expresión $2x^3 + 5x^2 - 4x + 10$
	\item Considera cada operador ($+, -, *, \wedge$) como una función binaria i.e. de dos parámetros. 
	\item Convierte cada término de la expresión usando las funciones binarias descritas en el paso anterior, recursivamente si es necesario.
		  Puedes usar funciones como argumentos de otras funciones.
		  Por ejemplo $2x^3$ es una $x$ \textsl{elevada} a la 3, y eso \textsl{multiplicado} por 2: $*(\wedge (x, 3), 2)$
	\item Transforma cada expresión a \textbf{notación polaca}, e.g. para expresar la suma aplicada a los argumentos 3 y 4, escribe $(+ 3,  4)$ en lugar de $+(3, 4)$
	\item Une cada término bajo la misma idea.
\end{enumerate}

Expresión de ejemplo:
\begin{tcolorbox}
	\begin{center}
		\texttt{(+ (- (+ (* 2 ($\wedge$ x 3)) (* 5 ($\wedge$ x 2))) (* 4 x)) 10)}
	\end{center}
\end{tcolorbox}

Puedes ahora generar un árbol con la expresión completa usando las reglas siguientes:

\begin{itemize}
	\item Lee la cadena de caracteres de izquierda a derecha.
	\item Al leer un \textbf{espacio en blanco}, considera que no escribirás más en el vértice actual. Continúa leyendo la cadena de caracteres.
	\item Al leer un \textbf{paréntesis que abre} `(', genera un vértice, que sea descendiente del último vértice visitado y muévete a él. Continúa leyendo.
	\item Al leer un \textbf{operador}, etiqueta el vértice actual con su símbolo y continúa leyendo.
	\item Al leer un \textbf{símbolo} (variable o número), genera un vértice que sea descendiente del último vértice visitado  y muévete a él. Etiqueta el vértice actual con dicho símbolo y regresa al vértice padre. Continúa leyendo.
	\item Al leer un \textbf{paréntesis que cierra} `)', regresa al padre del vértice en donde estás. Continúa leyendo.
	\item Deja de leer cuando no tengas más símbolos por leer en la cadena de caracteres. En este momento finaliza el proceso.
\end{itemize}

\pagebreak

Tomando en cuenta esta información, realiza los siguientes ejercicios.

\begin{enumerate}[label*=\alph*)]
	\item Genera el árbol de la expresión de ejemplo (10 \%)
	\item Propón dos polinomios de al menos 3 términos (4 \%)
	\begin{itemize}
		\item Genera la \textsl{S-expression} resultante para cada uno (30 \%)
		\item Genera el grafo de la \textsl{S-expression} resultante para cada uno (30 \%)
	\end{itemize}
	\item ¿Cuáles son los números máximos de vértices y de ejes que pueden existir en un solo término del polinomio? (6 \%)
	\item ¿Cuáles son los números máximos de de vértices y de ejes en una S-expression de $n$ términos?. Justifica tu respuesta con una prueba matemática (20 \%)
\end{enumerate}
\vspace{50ex}

% \section{Reflexión}

% Escribe los conceptos, tips o símbolos que consideres útiles para recordar lo visto en la sesión. Esta hoja te será de utilidad durante el examen.

\vfill

\textbf{Apegándome al Código de Ética de los Estudiantes del Tecnológico de Monterrey, me comprometo a que mi actuación en esta actividad esté regida por la honestidad académica.}

\end{document}